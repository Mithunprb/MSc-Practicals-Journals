\documentclass[11pt]{article}

    \usepackage[breakable]{tcolorbox}
    \usepackage{parskip} % Stop auto-indenting (to mimic markdown behaviour)
    
     \usepackage{tikz}
    % Basic figure setup, for now with no caption control since it's done
    % automatically by Pandoc (which extracts ![](path) syntax from Markdown).
    \usepackage{graphicx}
    % Maintain compatibility with old templates. Remove in nbconvert 6.0
    \let\Oldincludegraphics\includegraphics
    % Ensure that by default, figures have no caption (until we provide a
    % proper Figure object with a Caption API and a way to capture that
    % in the conversion process - todo).
    \usepackage{caption}
    \DeclareCaptionFormat{nocaption}{}
    \captionsetup{format=nocaption,aboveskip=0pt,belowskip=0pt}

    \usepackage{float}
    \floatplacement{figure}{H} % forces figures to be placed at the correct location
    \usepackage{xcolor} % Allow colors to be defined
    \usepackage{enumerate} % Needed for markdown enumerations to work
    \usepackage{geometry} % Used to adjust the document margins
    \usepackage{amsmath} % Equations
    \usepackage{amssymb} % Equations
    \usepackage{textcomp} % defines textquotesingle
    % Hack from http://tex.stackexchange.com/a/47451/13684:
    \AtBeginDocument{%
        \def\PYZsq{\textquotesingle}% Upright quotes in Pygmentized code
    }
    \usepackage{upquote} % Upright quotes for verbatim code
    \usepackage{eurosym} % defines \euro

    \usepackage{iftex}
    \ifPDFTeX
        \usepackage[T1]{fontenc}
        \IfFileExists{alphabeta.sty}{
              \usepackage{alphabeta}
          }{
              \usepackage[mathletters]{ucs}
              \usepackage[utf8x]{inputenc}
          }
    \else
        \usepackage{fontspec}
        \usepackage{unicode-math}
    \fi

    \usepackage{fancyvrb} % verbatim replacement that allows latex
    \usepackage{grffile} % extends the file name processing of package graphics 
                         % to support a larger range
    \makeatletter % fix for old versions of grffile with XeLaTeX
    \@ifpackagelater{grffile}{2019/11/01}
    {
      % Do nothing on new versions
    }
    {
      \def\Gread@@xetex#1{%
        \IfFileExists{"\Gin@base".bb}%
        {\Gread@eps{\Gin@base.bb}}%
        {\Gread@@xetex@aux#1}%
      }
    }
    \makeatother
    \usepackage[Export]{adjustbox} % Used to constrain images to a maximum size
    \adjustboxset{max size={0.9\linewidth}{0.9\paperheight}}

    % The hyperref package gives us a pdf with properly built
    % internal navigation ('pdf bookmarks' for the table of contents,
    % internal cross-reference links, web links for URLs, etc.)
    \usepackage{hyperref}
    % The default LaTeX title has an obnoxious amount of whitespace. By default,
    % titling removes some of it. It also provides customization options.
    \usepackage{titling}
    \usepackage{longtable} % longtable support required by pandoc >1.10
    \usepackage{booktabs}  % table support for pandoc > 1.12.2
    \usepackage{array}     % table support for pandoc >= 2.11.3
    \usepackage{calc}      % table minipage width calculation for pandoc >= 2.11.1
    \usepackage[inline]{enumitem} % IRkernel/repr support (it uses the enumerate* environment)
    \usepackage[normalem]{ulem} % ulem is needed to support strikethroughs (\sout)
                                % normalem makes italics be italics, not underlines
    \usepackage{mathrsfs}
    

    
    % Colors for the hyperref package
    \definecolor{urlcolor}{rgb}{0,.145,.698}
    \definecolor{linkcolor}{rgb}{.71,0.21,0.01}
    \definecolor{citecolor}{rgb}{.12,.54,.11}

    % ANSI colors
    \definecolor{ansi-black}{HTML}{3E424D}
    \definecolor{ansi-black-intense}{HTML}{282C36}
    \definecolor{ansi-red}{HTML}{E75C58}
    \definecolor{ansi-red-intense}{HTML}{B22B31}
    \definecolor{ansi-green}{HTML}{00A250}
    \definecolor{ansi-green-intense}{HTML}{007427}
    \definecolor{ansi-yellow}{HTML}{DDB62B}
    \definecolor{ansi-yellow-intense}{HTML}{B27D12}
    \definecolor{ansi-blue}{HTML}{208FFB}
    \definecolor{ansi-blue-intense}{HTML}{0065CA}
    \definecolor{ansi-magenta}{HTML}{D160C4}
    \definecolor{ansi-magenta-intense}{HTML}{A03196}
    \definecolor{ansi-cyan}{HTML}{60C6C8}
    \definecolor{ansi-cyan-intense}{HTML}{258F8F}
    \definecolor{ansi-white}{HTML}{C5C1B4}
    \definecolor{ansi-white-intense}{HTML}{A1A6B2}
    \definecolor{ansi-default-inverse-fg}{HTML}{FFFFFF}
    \definecolor{ansi-default-inverse-bg}{HTML}{000000}

    % common color for the border for error outputs.
    \definecolor{outerrorbackground}{HTML}{FFDFDF}

    % commands and environments needed by pandoc snippets
    % extracted from the output of `pandoc -s`
    \providecommand{\tightlist}{%
      \setlength{\itemsep}{0pt}\setlength{\parskip}{0pt}}
    \DefineVerbatimEnvironment{Highlighting}{Verbatim}{commandchars=\\\{\}}
    % Add ',fontsize=\small' for more characters per line
    \newenvironment{Shaded}{}{}
    \newcommand{\KeywordTok}[1]{\textcolor[rgb]{0.00,0.44,0.13}{\textbf{{#1}}}}
    \newcommand{\DataTypeTok}[1]{\textcolor[rgb]{0.56,0.13,0.00}{{#1}}}
    \newcommand{\DecValTok}[1]{\textcolor[rgb]{0.25,0.63,0.44}{{#1}}}
    \newcommand{\BaseNTok}[1]{\textcolor[rgb]{0.25,0.63,0.44}{{#1}}}
    \newcommand{\FloatTok}[1]{\textcolor[rgb]{0.25,0.63,0.44}{{#1}}}
    \newcommand{\CharTok}[1]{\textcolor[rgb]{0.25,0.44,0.63}{{#1}}}
    \newcommand{\StringTok}[1]{\textcolor[rgb]{0.25,0.44,0.63}{{#1}}}
    \newcommand{\CommentTok}[1]{\textcolor[rgb]{0.38,0.63,0.69}{\textit{{#1}}}}
    \newcommand{\OtherTok}[1]{\textcolor[rgb]{0.00,0.44,0.13}{{#1}}}
    \newcommand{\AlertTok}[1]{\textcolor[rgb]{1.00,0.00,0.00}{\textbf{{#1}}}}
    \newcommand{\FunctionTok}[1]{\textcolor[rgb]{0.02,0.16,0.49}{{#1}}}
    \newcommand{\RegionMarkerTok}[1]{{#1}}
    \newcommand{\ErrorTok}[1]{\textcolor[rgb]{1.00,0.00,0.00}{\textbf{{#1}}}}
    \newcommand{\NormalTok}[1]{{#1}}
    
    % Additional commands for more recent versions of Pandoc
    \newcommand{\ConstantTok}[1]{\textcolor[rgb]{0.53,0.00,0.00}{{#1}}}
    \newcommand{\SpecialCharTok}[1]{\textcolor[rgb]{0.25,0.44,0.63}{{#1}}}
    \newcommand{\VerbatimStringTok}[1]{\textcolor[rgb]{0.25,0.44,0.63}{{#1}}}
    \newcommand{\SpecialStringTok}[1]{\textcolor[rgb]{0.73,0.40,0.53}{{#1}}}
    \newcommand{\ImportTok}[1]{{#1}}
    \newcommand{\DocumentationTok}[1]{\textcolor[rgb]{0.73,0.13,0.13}{\textit{{#1}}}}
    \newcommand{\AnnotationTok}[1]{\textcolor[rgb]{0.38,0.63,0.69}{\textbf{\textit{{#1}}}}}
    \newcommand{\CommentVarTok}[1]{\textcolor[rgb]{0.38,0.63,0.69}{\textbf{\textit{{#1}}}}}
    \newcommand{\VariableTok}[1]{\textcolor[rgb]{0.10,0.09,0.49}{{#1}}}
    \newcommand{\ControlFlowTok}[1]{\textcolor[rgb]{0.00,0.44,0.13}{\textbf{{#1}}}}
    \newcommand{\OperatorTok}[1]{\textcolor[rgb]{0.40,0.40,0.40}{{#1}}}
    \newcommand{\BuiltInTok}[1]{{#1}}
    \newcommand{\ExtensionTok}[1]{{#1}}
    \newcommand{\PreprocessorTok}[1]{\textcolor[rgb]{0.74,0.48,0.00}{{#1}}}
    \newcommand{\AttributeTok}[1]{\textcolor[rgb]{0.49,0.56,0.16}{{#1}}}
    \newcommand{\InformationTok}[1]{\textcolor[rgb]{0.38,0.63,0.69}{\textbf{\textit{{#1}}}}}
    \newcommand{\WarningTok}[1]{\textcolor[rgb]{0.38,0.63,0.69}{\textbf{\textit{{#1}}}}}
    
    
    % Define a nice break command that doesn't care if a line doesn't already
    % exist.
    \def\br{\hspace*{\fill} \\* }
    % Math Jax compatibility definitions
    \def\gt{>}
    \def\lt{<}
    \let\Oldtex\TeX
    \let\Oldlatex\LaTeX
    \renewcommand{\TeX}{\textrm{\Oldtex}}
    \renewcommand{\LaTeX}{\textrm{\Oldlatex}}
    % Document parameters
    % Document title
    \title{OR-Journal}
    
    
    
    
    
% Pygments definitions
\makeatletter
\def\PY@reset{\let\PY@it=\relax \let\PY@bf=\relax%
    \let\PY@ul=\relax \let\PY@tc=\relax%
    \let\PY@bc=\relax \let\PY@ff=\relax}
\def\PY@tok#1{\csname PY@tok@#1\endcsname}
\def\PY@toks#1+{\ifx\relax#1\empty\else%
    \PY@tok{#1}\expandafter\PY@toks\fi}
\def\PY@do#1{\PY@bc{\PY@tc{\PY@ul{%
    \PY@it{\PY@bf{\PY@ff{#1}}}}}}}
\def\PY#1#2{\PY@reset\PY@toks#1+\relax+\PY@do{#2}}

\@namedef{PY@tok@w}{\def\PY@tc##1{\textcolor[rgb]{0.73,0.73,0.73}{##1}}}
\@namedef{PY@tok@c}{\let\PY@it=\textit\def\PY@tc##1{\textcolor[rgb]{0.24,0.48,0.48}{##1}}}
\@namedef{PY@tok@cp}{\def\PY@tc##1{\textcolor[rgb]{0.61,0.40,0.00}{##1}}}
\@namedef{PY@tok@k}{\let\PY@bf=\textbf\def\PY@tc##1{\textcolor[rgb]{0.00,0.50,0.00}{##1}}}
\@namedef{PY@tok@kp}{\def\PY@tc##1{\textcolor[rgb]{0.00,0.50,0.00}{##1}}}
\@namedef{PY@tok@kt}{\def\PY@tc##1{\textcolor[rgb]{0.69,0.00,0.25}{##1}}}
\@namedef{PY@tok@o}{\def\PY@tc##1{\textcolor[rgb]{0.40,0.40,0.40}{##1}}}
\@namedef{PY@tok@ow}{\let\PY@bf=\textbf\def\PY@tc##1{\textcolor[rgb]{0.67,0.13,1.00}{##1}}}
\@namedef{PY@tok@nb}{\def\PY@tc##1{\textcolor[rgb]{0.00,0.50,0.00}{##1}}}
\@namedef{PY@tok@nf}{\def\PY@tc##1{\textcolor[rgb]{0.00,0.00,1.00}{##1}}}
\@namedef{PY@tok@nc}{\let\PY@bf=\textbf\def\PY@tc##1{\textcolor[rgb]{0.00,0.00,1.00}{##1}}}
\@namedef{PY@tok@nn}{\let\PY@bf=\textbf\def\PY@tc##1{\textcolor[rgb]{0.00,0.00,1.00}{##1}}}
\@namedef{PY@tok@ne}{\let\PY@bf=\textbf\def\PY@tc##1{\textcolor[rgb]{0.80,0.25,0.22}{##1}}}
\@namedef{PY@tok@nv}{\def\PY@tc##1{\textcolor[rgb]{0.10,0.09,0.49}{##1}}}
\@namedef{PY@tok@no}{\def\PY@tc##1{\textcolor[rgb]{0.53,0.00,0.00}{##1}}}
\@namedef{PY@tok@nl}{\def\PY@tc##1{\textcolor[rgb]{0.46,0.46,0.00}{##1}}}
\@namedef{PY@tok@ni}{\let\PY@bf=\textbf\def\PY@tc##1{\textcolor[rgb]{0.44,0.44,0.44}{##1}}}
\@namedef{PY@tok@na}{\def\PY@tc##1{\textcolor[rgb]{0.41,0.47,0.13}{##1}}}
\@namedef{PY@tok@nt}{\let\PY@bf=\textbf\def\PY@tc##1{\textcolor[rgb]{0.00,0.50,0.00}{##1}}}
\@namedef{PY@tok@nd}{\def\PY@tc##1{\textcolor[rgb]{0.67,0.13,1.00}{##1}}}
\@namedef{PY@tok@s}{\def\PY@tc##1{\textcolor[rgb]{0.73,0.13,0.13}{##1}}}
\@namedef{PY@tok@sd}{\let\PY@it=\textit\def\PY@tc##1{\textcolor[rgb]{0.73,0.13,0.13}{##1}}}
\@namedef{PY@tok@si}{\let\PY@bf=\textbf\def\PY@tc##1{\textcolor[rgb]{0.64,0.35,0.47}{##1}}}
\@namedef{PY@tok@se}{\let\PY@bf=\textbf\def\PY@tc##1{\textcolor[rgb]{0.67,0.36,0.12}{##1}}}
\@namedef{PY@tok@sr}{\def\PY@tc##1{\textcolor[rgb]{0.64,0.35,0.47}{##1}}}
\@namedef{PY@tok@ss}{\def\PY@tc##1{\textcolor[rgb]{0.10,0.09,0.49}{##1}}}
\@namedef{PY@tok@sx}{\def\PY@tc##1{\textcolor[rgb]{0.00,0.50,0.00}{##1}}}
\@namedef{PY@tok@m}{\def\PY@tc##1{\textcolor[rgb]{0.40,0.40,0.40}{##1}}}
\@namedef{PY@tok@gh}{\let\PY@bf=\textbf\def\PY@tc##1{\textcolor[rgb]{0.00,0.00,0.50}{##1}}}
\@namedef{PY@tok@gu}{\let\PY@bf=\textbf\def\PY@tc##1{\textcolor[rgb]{0.50,0.00,0.50}{##1}}}
\@namedef{PY@tok@gd}{\def\PY@tc##1{\textcolor[rgb]{0.63,0.00,0.00}{##1}}}
\@namedef{PY@tok@gi}{\def\PY@tc##1{\textcolor[rgb]{0.00,0.52,0.00}{##1}}}
\@namedef{PY@tok@gr}{\def\PY@tc##1{\textcolor[rgb]{0.89,0.00,0.00}{##1}}}
\@namedef{PY@tok@ge}{\let\PY@it=\textit}
\@namedef{PY@tok@gs}{\let\PY@bf=\textbf}
\@namedef{PY@tok@gp}{\let\PY@bf=\textbf\def\PY@tc##1{\textcolor[rgb]{0.00,0.00,0.50}{##1}}}
\@namedef{PY@tok@go}{\def\PY@tc##1{\textcolor[rgb]{0.44,0.44,0.44}{##1}}}
\@namedef{PY@tok@gt}{\def\PY@tc##1{\textcolor[rgb]{0.00,0.27,0.87}{##1}}}
\@namedef{PY@tok@err}{\def\PY@bc##1{{\setlength{\fboxsep}{\string -\fboxrule}\fcolorbox[rgb]{1.00,0.00,0.00}{1,1,1}{\strut ##1}}}}
\@namedef{PY@tok@kc}{\let\PY@bf=\textbf\def\PY@tc##1{\textcolor[rgb]{0.00,0.50,0.00}{##1}}}
\@namedef{PY@tok@kd}{\let\PY@bf=\textbf\def\PY@tc##1{\textcolor[rgb]{0.00,0.50,0.00}{##1}}}
\@namedef{PY@tok@kn}{\let\PY@bf=\textbf\def\PY@tc##1{\textcolor[rgb]{0.00,0.50,0.00}{##1}}}
\@namedef{PY@tok@kr}{\let\PY@bf=\textbf\def\PY@tc##1{\textcolor[rgb]{0.00,0.50,0.00}{##1}}}
\@namedef{PY@tok@bp}{\def\PY@tc##1{\textcolor[rgb]{0.00,0.50,0.00}{##1}}}
\@namedef{PY@tok@fm}{\def\PY@tc##1{\textcolor[rgb]{0.00,0.00,1.00}{##1}}}
\@namedef{PY@tok@vc}{\def\PY@tc##1{\textcolor[rgb]{0.10,0.09,0.49}{##1}}}
\@namedef{PY@tok@vg}{\def\PY@tc##1{\textcolor[rgb]{0.10,0.09,0.49}{##1}}}
\@namedef{PY@tok@vi}{\def\PY@tc##1{\textcolor[rgb]{0.10,0.09,0.49}{##1}}}
\@namedef{PY@tok@vm}{\def\PY@tc##1{\textcolor[rgb]{0.10,0.09,0.49}{##1}}}
\@namedef{PY@tok@sa}{\def\PY@tc##1{\textcolor[rgb]{0.73,0.13,0.13}{##1}}}
\@namedef{PY@tok@sb}{\def\PY@tc##1{\textcolor[rgb]{0.73,0.13,0.13}{##1}}}
\@namedef{PY@tok@sc}{\def\PY@tc##1{\textcolor[rgb]{0.73,0.13,0.13}{##1}}}
\@namedef{PY@tok@dl}{\def\PY@tc##1{\textcolor[rgb]{0.73,0.13,0.13}{##1}}}
\@namedef{PY@tok@s2}{\def\PY@tc##1{\textcolor[rgb]{0.73,0.13,0.13}{##1}}}
\@namedef{PY@tok@sh}{\def\PY@tc##1{\textcolor[rgb]{0.73,0.13,0.13}{##1}}}
\@namedef{PY@tok@s1}{\def\PY@tc##1{\textcolor[rgb]{0.73,0.13,0.13}{##1}}}
\@namedef{PY@tok@mb}{\def\PY@tc##1{\textcolor[rgb]{0.40,0.40,0.40}{##1}}}
\@namedef{PY@tok@mf}{\def\PY@tc##1{\textcolor[rgb]{0.40,0.40,0.40}{##1}}}
\@namedef{PY@tok@mh}{\def\PY@tc##1{\textcolor[rgb]{0.40,0.40,0.40}{##1}}}
\@namedef{PY@tok@mi}{\def\PY@tc##1{\textcolor[rgb]{0.40,0.40,0.40}{##1}}}
\@namedef{PY@tok@il}{\def\PY@tc##1{\textcolor[rgb]{0.40,0.40,0.40}{##1}}}
\@namedef{PY@tok@mo}{\def\PY@tc##1{\textcolor[rgb]{0.40,0.40,0.40}{##1}}}
\@namedef{PY@tok@ch}{\let\PY@it=\textit\def\PY@tc##1{\textcolor[rgb]{0.24,0.48,0.48}{##1}}}
\@namedef{PY@tok@cm}{\let\PY@it=\textit\def\PY@tc##1{\textcolor[rgb]{0.24,0.48,0.48}{##1}}}
\@namedef{PY@tok@cpf}{\let\PY@it=\textit\def\PY@tc##1{\textcolor[rgb]{0.24,0.48,0.48}{##1}}}
\@namedef{PY@tok@c1}{\let\PY@it=\textit\def\PY@tc##1{\textcolor[rgb]{0.24,0.48,0.48}{##1}}}
\@namedef{PY@tok@cs}{\let\PY@it=\textit\def\PY@tc##1{\textcolor[rgb]{0.24,0.48,0.48}{##1}}}

\def\PYZbs{\char`\\}
\def\PYZus{\char`\_}
\def\PYZob{\char`\{}
\def\PYZcb{\char`\}}
\def\PYZca{\char`\^}
\def\PYZam{\char`\&}
\def\PYZlt{\char`\<}
\def\PYZgt{\char`\>}
\def\PYZsh{\char`\#}
\def\PYZpc{\char`\%}
\def\PYZdl{\char`\$}
\def\PYZhy{\char`\-}
\def\PYZsq{\char`\'}
\def\PYZdq{\char`\"}
\def\PYZti{\char`\~}
% for compatibility with earlier versions
\def\PYZat{@}
\def\PYZlb{[}
\def\PYZrb{]}
\makeatother


    % For linebreaks inside Verbatim environment from package fancyvrb. 
    \makeatletter
        \newbox\Wrappedcontinuationbox 
        \newbox\Wrappedvisiblespacebox 
        \newcommand*\Wrappedvisiblespace {\textcolor{red}{\textvisiblespace}} 
        \newcommand*\Wrappedcontinuationsymbol {\textcolor{red}{\llap{\tiny$\m@th\hookrightarrow$}}} 
        \newcommand*\Wrappedcontinuationindent {3ex } 
        \newcommand*\Wrappedafterbreak {\kern\Wrappedcontinuationindent\copy\Wrappedcontinuationbox} 
        % Take advantage of the already applied Pygments mark-up to insert 
        % potential linebreaks for TeX processing. 
        %        {, <, #, %, $, ' and ": go to next line. 
        %        _, }, ^, &, >, - and ~: stay at end of broken line. 
        % Use of \textquotesingle for straight quote. 
        \newcommand*\Wrappedbreaksatspecials {% 
            \def\PYGZus{\discretionary{\char`\_}{\Wrappedafterbreak}{\char`\_}}% 
            \def\PYGZob{\discretionary{}{\Wrappedafterbreak\char`\{}{\char`\{}}% 
            \def\PYGZcb{\discretionary{\char`\}}{\Wrappedafterbreak}{\char`\}}}% 
            \def\PYGZca{\discretionary{\char`\^}{\Wrappedafterbreak}{\char`\^}}% 
            \def\PYGZam{\discretionary{\char`\&}{\Wrappedafterbreak}{\char`\&}}% 
            \def\PYGZlt{\discretionary{}{\Wrappedafterbreak\char`\<}{\char`\<}}% 
            \def\PYGZgt{\discretionary{\char`\>}{\Wrappedafterbreak}{\char`\>}}% 
            \def\PYGZsh{\discretionary{}{\Wrappedafterbreak\char`\#}{\char`\#}}% 
            \def\PYGZpc{\discretionary{}{\Wrappedafterbreak\char`\%}{\char`\%}}% 
            \def\PYGZdl{\discretionary{}{\Wrappedafterbreak\char`\$}{\char`\$}}% 
            \def\PYGZhy{\discretionary{\char`\-}{\Wrappedafterbreak}{\char`\-}}% 
            \def\PYGZsq{\discretionary{}{\Wrappedafterbreak\textquotesingle}{\textquotesingle}}% 
            \def\PYGZdq{\discretionary{}{\Wrappedafterbreak\char`\"}{\char`\"}}% 
            \def\PYGZti{\discretionary{\char`\~}{\Wrappedafterbreak}{\char`\~}}% 
        } 
        % Some characters . , ; ? ! / are not pygmentized. 
        % This macro makes them "active" and they will insert potential linebreaks 
        \newcommand*\Wrappedbreaksatpunct {% 
            \lccode`\~`\.\lowercase{\def~}{\discretionary{\hbox{\char`\.}}{\Wrappedafterbreak}{\hbox{\char`\.}}}% 
            \lccode`\~`\,\lowercase{\def~}{\discretionary{\hbox{\char`\,}}{\Wrappedafterbreak}{\hbox{\char`\,}}}% 
            \lccode`\~`\;\lowercase{\def~}{\discretionary{\hbox{\char`\;}}{\Wrappedafterbreak}{\hbox{\char`\;}}}% 
            \lccode`\~`\:\lowercase{\def~}{\discretionary{\hbox{\char`\:}}{\Wrappedafterbreak}{\hbox{\char`\:}}}% 
            \lccode`\~`\?\lowercase{\def~}{\discretionary{\hbox{\char`\?}}{\Wrappedafterbreak}{\hbox{\char`\?}}}% 
            \lccode`\~`\!\lowercase{\def~}{\discretionary{\hbox{\char`\!}}{\Wrappedafterbreak}{\hbox{\char`\!}}}% 
            \lccode`\~`\/\lowercase{\def~}{\discretionary{\hbox{\char`\/}}{\Wrappedafterbreak}{\hbox{\char`\/}}}% 
            \catcode`\.\active
            \catcode`\,\active 
            \catcode`\;\active
            \catcode`\:\active
            \catcode`\?\active
            \catcode`\!\active
            \catcode`\/\active 
            \lccode`\~`\~ 	
        }
    \makeatother

    \let\OriginalVerbatim=\Verbatim
    \makeatletter
    \renewcommand{\Verbatim}[1][1]{%
        %\parskip\z@skip
        \sbox\Wrappedcontinuationbox {\Wrappedcontinuationsymbol}%
        \sbox\Wrappedvisiblespacebox {\FV@SetupFont\Wrappedvisiblespace}%
        \def\FancyVerbFormatLine ##1{\hsize\linewidth
            \vtop{\raggedright\hyphenpenalty\z@\exhyphenpenalty\z@
                \doublehyphendemerits\z@\finalhyphendemerits\z@
                \strut ##1\strut}%
        }%
        % If the linebreak is at a space, the latter will be displayed as visible
        % space at end of first line, and a continuation symbol starts next line.
        % Stretch/shrink are however usually zero for typewriter font.
        \def\FV@Space {%
            \nobreak\hskip\z@ plus\fontdimen3\font minus\fontdimen4\font
            \discretionary{\copy\Wrappedvisiblespacebox}{\Wrappedafterbreak}
            {\kern\fontdimen2\font}%
        }%
        
        % Allow breaks at special characters using \PYG... macros.
        \Wrappedbreaksatspecials
        % Breaks at punctuation characters . , ; ? ! and / need catcode=\active 	
        \OriginalVerbatim[#1,codes*=\Wrappedbreaksatpunct]%
    }
    \makeatother

    % Exact colors from NB
    \definecolor{incolor}{HTML}{303F9F}
    \definecolor{outcolor}{HTML}{D84315}
    \definecolor{cellborder}{HTML}{CFCFCF}
    \definecolor{cellbackground}{HTML}{F7F7F7}
    
    % prompt
    \makeatletter
    \newcommand{\boxspacing}{\kern\kvtcb@left@rule\kern\kvtcb@boxsep}
    \makeatother
    \newcommand{\prompt}[4]{
        {\ttfamily\llap{{\color{#2}[#3]:\hspace{3pt}#4}}\vspace{-\baselineskip}}
    }
    

    
    % Prevent overflowing lines due to hard-to-break entities
    \sloppy 
    % Setup hyperref package
    \hypersetup{
      breaklinks=true,  % so long urls are correctly broken across lines
      colorlinks=true,
      urlcolor=urlcolor,
      linkcolor=linkcolor,
      citecolor=citecolor,
      }
    % Slightly bigger margins than the latex defaults
    
    \geometry{verbose,tmargin=1in,bmargin=1in,lmargin=1in,rmargin=1in}
    
        
    \title{\huge{\textbf{ Operation Research Journal}} \\
    \LARGE{M.Sc Part II Computer Science}}
    \author{Mithun Parab 509}    

\begin{document}
    
    % \maketitle
    \pagenumbering{roman} % Start roman numbering
\clearpage\maketitle
\thispagestyle{empty}
\begin{center}
    \begin{figure}[h]
        \centering
        \includegraphics[width=7cm]{RJCLG.png}
        %\caption{Your caption here}
        \label{fig:logo}
    \end{figure}

    \large{R.J. College of Arts, Science \& Commerce \\
    Operation Research \\
    Seat number: 509
    }
\end{center}
\newpage
\frenchspacing
    \begin{tikzpicture}[remember picture,overlay]
    \node[anchor=north west,yshift=-10pt,xshift=1pt]%
        at (current page.north west)
        {\includegraphics[width=3cm, height=3cm]{RJCLG.png}};
            \end{tikzpicture}
            \begin{center}
               \textbf{\Large{ \uppercase{
     Ramniranjan Jhunjhunwala College
     }}}\\  
    \textbf{\large{Ghatkopar (W), Mumbai-400 086}}
    % Vidyavihar, Mumbai-400 077\\
    % Autonomous- Affiliated to University of Mumbai\\
    % Department of Information Technology
            \end{center}


    
       \begin{tikzpicture}[remember picture,overlay]
    \node[anchor=north east,yshift=-19pt,xshift=1pt]%
        at (current page.north east)
        {\includegraphics[width=3cm, height=1.5cm]{RJC.png}};
    \end{tikzpicture}
    
\begin{center}

    \end{center}
    \begin{center}
    \textbf{ \uppercase{ \Large{CERTIFICATE} \\}}
     \hspace{0.22cm} M.Sc Part II \\
    \hspace{0.22cm} Computer Science \\
    \hspace{0.22cm} 2022-2023 \\
    \end{center}
    \vspace{1cm}
    \begin{center}
            \begin{minipage}{42em}
    \hspace{1.5cm} This is to certify that \textbf{Mithun Sahdev Parab} of M.Sc Prat II (Sem-III) Computer Science, Seat No \textbf{509} of satisfactorily completed the practicals of \textbf{\uppercase{Operation Research (Paper III)}} during the academic year \textbf{2023 \ - \ 2024} as specified by the \textbf{\uppercase{Mumbai University}}.
    \end{minipage}
    \end{center}
\hspace{0.7cm}
\\\\
No. of Experiments completed \hspace{0.25cm} \textbf{10} \hspace{0.25cm} out of \hspace{0.25cm} \textbf{10} \hspace{0.25cm}
    \\\\
    \textbf{
    Sign of Incharge: \\
    \hspace{1.5cm}Date: \today        \hspace{5.6cm} \textbf{Seat Number: 509}
    \\\\
    Sign of Examiner: \\
    \hspace{1.5cm}Date:  \today			\hspace{5.5cm}			 Course Co-ordinator}
    

\newpage
\tableofcontents
\begin{center}
    Link for \href{https://github.com/Mithunprb/MSc-Practicals-Journals/tree/main/MSc-Part2/OR/Practical}{GitHub}
\end{center}
\newpage
\pagenumbering{arabic} % Start roman numbering
    
    

    
    \section{Practical 01: \uppercase {Graphical method using R
programming}}\label{practical-01-graphical-method-using-r-programming}

\subsection{Find a geometrical interpretation and solution as well for
the following LP
problem}\label{find-a-geometrical-interpretation-and-solution-as-well-for-the-following-lp-problem}

\(Max \ z= 3x1 + 5x2\)

subject to constraints:

\(x1 + 2x2<=2000\)

\(x1+x2<=1500\)

\(x2<=600\)

\(x1,x2>=0\)

    \begin{tcolorbox}[breakable, size=fbox, boxrule=1pt, pad at break*=1mm,colback=cellbackground, colframe=cellborder]
\prompt{In}{incolor}{ }{\boxspacing}
\begin{Verbatim}[commandchars=\\\{\}]
\PY{n+nf}{install.packages}\PY{p}{(}\PY{l+s}{\PYZdq{}}\PY{l+s}{lpSolve\PYZdq{}}\PY{p}{)}
\end{Verbatim}
\end{tcolorbox}

    \begin{Verbatim}[commandchars=\\\{\}]
Installing package into ‘/usr/local/lib/R/site-library’
(as ‘lib’ is unspecified)

    \end{Verbatim}

    \begin{tcolorbox}[breakable, size=fbox, boxrule=1pt, pad at break*=1mm,colback=cellbackground, colframe=cellborder]
\prompt{In}{incolor}{ }{\boxspacing}
\begin{Verbatim}[commandchars=\\\{\}]
\PY{c+c1}{\PYZsh{} R Program}
\PY{c+c1}{\PYZsh{}Find a geometrical interpretation and solution as well for the following LP problem}
\PY{c+c1}{\PYZsh{}Max z= 3x1 + 5x2}
\PY{c+c1}{\PYZsh{}subject to constraints:}
\PY{c+c1}{\PYZsh{}x1+2x2\PYZlt{}=2000}
\PY{c+c1}{\PYZsh{}x1+x2\PYZlt{}=1500}
\PY{c+c1}{\PYZsh{}x2\PYZlt{}=600}
\PY{c+c1}{\PYZsh{}x1,x2\PYZgt{}=0}
\PY{c+c1}{\PYZsh{} Load lpSolve}
\PY{n+nf}{require}\PY{p}{(}\PY{n}{lpSolve}\PY{p}{)}
\end{Verbatim}
\end{tcolorbox}

    \begin{Verbatim}[commandchars=\\\{\}]
Loading required package: lpSolve

    \end{Verbatim}

    \begin{tcolorbox}[breakable, size=fbox, boxrule=1pt, pad at break*=1mm,colback=cellbackground, colframe=cellborder]
\prompt{In}{incolor}{ }{\boxspacing}
\begin{Verbatim}[commandchars=\\\{\}]
\PY{c+c1}{\PYZsh{}\PYZsh{} Set the coefficients of the decision variables \PYZhy{}\PYZgt{} C of objective function}
\PY{n}{C}\PY{+w}{ }\PY{o}{\PYZlt{}\PYZhy{}}\PY{+w}{ }\PY{n+nf}{c}\PY{p}{(}\PY{l+m}{3}\PY{p}{,}\PY{l+m}{5}\PY{p}{)}
\PY{c+c1}{\PYZsh{} Create constraint martix B}
\PY{n}{A}\PY{+w}{ }\PY{o}{\PYZlt{}\PYZhy{}}\PY{+w}{ }\PY{n+nf}{matrix}\PY{p}{(}\PY{n+nf}{c}\PY{p}{(}\PY{l+m}{1}\PY{p}{,}\PY{+w}{ }\PY{l+m}{2}\PY{p}{,}
\PY{+w}{              }\PY{l+m}{1}\PY{p}{,}\PY{+w}{ }\PY{l+m}{1}\PY{p}{,}
\PY{+w}{              }\PY{l+m}{0}\PY{p}{,}\PY{+w}{ }\PY{l+m}{1}
\PY{+w}{              }\PY{p}{)}\PY{p}{,}\PY{+w}{ }\PY{n}{nrow}\PY{o}{=}\PY{l+m}{3}\PY{p}{,}\PY{+w}{ }\PY{n}{byrow}\PY{o}{=}\PY{k+kc}{TRUE}\PY{p}{)}
\PY{c+c1}{\PYZsh{} Right hand side for the constraints}
\PY{n}{B}\PY{+w}{ }\PY{o}{\PYZlt{}\PYZhy{}}\PY{+w}{ }\PY{n+nf}{c}\PY{p}{(}\PY{l+m}{2000}\PY{p}{,}\PY{l+m}{1500}\PY{p}{,}\PY{l+m}{600}\PY{p}{)}
\end{Verbatim}
\end{tcolorbox}

    \begin{tcolorbox}[breakable, size=fbox, boxrule=1pt, pad at break*=1mm,colback=cellbackground, colframe=cellborder]
\prompt{In}{incolor}{ }{\boxspacing}
\begin{Verbatim}[commandchars=\\\{\}]
\PY{c+c1}{\PYZsh{} Direction of the constraints}
\PY{n}{sconstranints\PYZus{}direction}\PY{+w}{ }\PY{o}{\PYZlt{}\PYZhy{}}\PY{+w}{ }\PY{n+nf}{c}\PY{p}{(}\PY{l+s}{\PYZdq{}}\PY{l+s}{\PYZlt{}=\PYZdq{}}\PY{p}{,}\PY{+w}{ }\PY{l+s}{\PYZdq{}}\PY{l+s}{\PYZlt{}=\PYZdq{}}\PY{p}{,}\PY{+w}{ }\PY{l+s}{\PYZdq{}}\PY{l+s}{\PYZlt{}=\PYZdq{}}\PY{p}{)}
\end{Verbatim}
\end{tcolorbox}

    \begin{tcolorbox}[breakable, size=fbox, boxrule=1pt, pad at break*=1mm,colback=cellbackground, colframe=cellborder]
\prompt{In}{incolor}{ }{\boxspacing}
\begin{Verbatim}[commandchars=\\\{\}]
\PY{c+c1}{\PYZsh{} Create empty example plot}
\PY{n+nf}{plot.new}\PY{p}{(}\PY{p}{)}
\PY{n+nf}{plot.window}\PY{p}{(}\PY{n}{xlim}\PY{o}{=}\PY{n+nf}{c}\PY{p}{(}\PY{l+m}{0}\PY{p}{,}\PY{l+m}{2000}\PY{p}{)}\PY{p}{,}\PY{+w}{ }\PY{n}{ylim}\PY{o}{=}\PY{n+nf}{c}\PY{p}{(}\PY{l+m}{0}\PY{p}{,}\PY{l+m}{2000}\PY{p}{)}\PY{p}{)}
\PY{n+nf}{axis}\PY{p}{(}\PY{l+m}{1}\PY{p}{)}
\PY{n+nf}{axis}\PY{p}{(}\PY{l+m}{2}\PY{p}{)}
\PY{n+nf}{title}\PY{p}{(}\PY{n}{main}\PY{o}{=}\PY{l+s}{\PYZdq{}}\PY{l+s}{LPP using Graphical method\PYZdq{}}\PY{p}{)}
\PY{n+nf}{title}\PY{p}{(}\PY{n}{xlab}\PY{o}{=}\PY{l+s}{\PYZdq{}}\PY{l+s}{X axis\PYZdq{}}\PY{p}{)}
\PY{n+nf}{title}\PY{p}{(}\PY{n}{ylab}\PY{o}{=}\PY{l+s}{\PYZdq{}}\PY{l+s}{Y axis\PYZdq{}}\PY{p}{)}
\PY{n+nf}{box}\PY{p}{(}\PY{p}{)}
\PY{c+c1}{\PYZsh{} Draw one line}
\PY{n+nf}{segments}\PY{p}{(}\PY{n}{x0}\PY{+w}{ }\PY{o}{=}\PY{+w}{ }\PY{l+m}{2000}\PY{p}{,}\PY{+w}{ }\PY{n}{y0}\PY{+w}{ }\PY{o}{=}\PY{+w}{ }\PY{l+m}{0}\PY{p}{,}\PY{+w}{ }\PY{n}{x1}\PY{+w}{ }\PY{o}{=}\PY{+w}{ }\PY{l+m}{0}\PY{p}{,}\PY{+w}{ }\PY{n}{y1}\PY{+w}{ }\PY{o}{=}\PY{+w}{ }\PY{l+m}{1000}\PY{p}{,}\PY{+w}{ }\PY{n}{col}\PY{+w}{ }\PY{o}{=}\PY{+w}{ }\PY{l+s}{\PYZdq{}}\PY{l+s}{green\PYZdq{}}\PY{p}{)}
\PY{n+nf}{segments}\PY{p}{(}\PY{n}{x0}\PY{+w}{ }\PY{o}{=}\PY{+w}{ }\PY{l+m}{1500}\PY{p}{,}\PY{+w}{ }\PY{n}{y0}\PY{+w}{ }\PY{o}{=}\PY{+w}{ }\PY{l+m}{0}\PY{p}{,}\PY{+w}{ }\PY{n}{x1}\PY{+w}{ }\PY{o}{=}\PY{+w}{ }\PY{l+m}{0}\PY{p}{,}\PY{+w}{ }\PY{n}{y1}\PY{+w}{ }\PY{o}{=}\PY{+w}{ }\PY{l+m}{1500}\PY{p}{,}\PY{+w}{ }\PY{n}{col}\PY{+w}{ }\PY{o}{=}\PY{+w}{ }\PY{l+s}{\PYZdq{}}\PY{l+s}{green\PYZdq{}}\PY{p}{)}
\PY{n+nf}{segments}\PY{p}{(}\PY{n}{x0}\PY{+w}{ }\PY{o}{=}\PY{+w}{ }\PY{l+m}{0}\PY{p}{,}\PY{+w}{ }\PY{n}{y0}\PY{+w}{ }\PY{o}{=}\PY{+w}{ }\PY{l+m}{0}\PY{p}{,}\PY{+w}{ }\PY{n}{x1}\PY{+w}{ }\PY{o}{=}\PY{+w}{ }\PY{l+m}{600}\PY{p}{,}\PY{+w}{ }\PY{n}{y1}\PY{+w}{ }\PY{o}{=}\PY{+w}{ }\PY{l+m}{0}\PY{p}{,}\PY{+w}{ }\PY{n}{col}\PY{+w}{ }\PY{o}{=}\PY{+w}{ }\PY{l+s}{\PYZdq{}}\PY{l+s}{green\PYZdq{}}\PY{p}{)}
\end{Verbatim}
\end{tcolorbox}

    \begin{center}
    \adjustimage{max size={0.9\linewidth}{0.9\paperheight}}{output_5_0.png}
    \end{center}
    { \hspace*{\fill} \\}
    
    \begin{tcolorbox}[breakable, size=fbox, boxrule=1pt, pad at break*=1mm,colback=cellbackground, colframe=cellborder]
\prompt{In}{incolor}{ }{\boxspacing}
\begin{Verbatim}[commandchars=\\\{\}]
\PY{c+c1}{\PYZsh{} Find the optimal solution}
\PY{n}{optimum}\PY{+w}{ }\PY{o}{\PYZlt{}\PYZhy{}}\PY{+w}{ }\PY{n+nf}{lp}\PY{p}{(}\PY{n}{direction}\PY{o}{=}\PY{l+s}{\PYZdq{}}\PY{l+s}{max\PYZdq{}}\PY{p}{,}
\PY{n}{objective.in}\PY{+w}{ }\PY{o}{=}\PY{+w}{ }\PY{n}{C}\PY{p}{,}
\PY{n}{const.mat}\PY{+w}{ }\PY{o}{=}\PY{+w}{ }\PY{n}{A}\PY{p}{,}
\PY{n}{const.dir}\PY{+w}{ }\PY{o}{=}\PY{+w}{ }\PY{n}{sconstranints\PYZus{}direction}\PY{p}{,}
\PY{n}{const.rhs}\PY{+w}{ }\PY{o}{=}\PY{+w}{ }\PY{n}{B}\PY{p}{,}
\PY{n}{all.int}\PY{+w}{ }\PY{o}{=}\PY{+w}{ }\PY{n+nb+bp}{T}\PY{p}{)}
\PY{c+c1}{\PYZsh{} Print status: 0 = success, 2 = no feasible solution}
\PY{n+nf}{print}\PY{p}{(}\PY{n}{optimum}\PY{o}{\PYZdl{}}\PY{n}{status}\PY{p}{)}
\end{Verbatim}
\end{tcolorbox}

    \begin{Verbatim}[commandchars=\\\{\}]
[1] 0
    \end{Verbatim}

    \begin{tcolorbox}[breakable, size=fbox, boxrule=1pt, pad at break*=1mm,colback=cellbackground, colframe=cellborder]
\prompt{In}{incolor}{ }{\boxspacing}
\begin{Verbatim}[commandchars=\\\{\}]
\PY{c+c1}{\PYZsh{} Display the optimum values for x1,x2}
\PY{n}{best\PYZus{}sol}\PY{+w}{ }\PY{o}{\PYZlt{}\PYZhy{}}\PY{+w}{ }\PY{n}{optimum}\PY{o}{\PYZdl{}}\PY{n}{solution}
\PY{n+nf}{names}\PY{p}{(}\PY{n}{best\PYZus{}sol}\PY{p}{)}\PY{+w}{ }\PY{o}{\PYZlt{}\PYZhy{}}\PY{+w}{ }\PY{n+nf}{c}\PY{p}{(}\PY{l+s}{\PYZdq{}}\PY{l+s}{x1\PYZdq{}}\PY{p}{,}\PY{+w}{ }\PY{l+s}{\PYZdq{}}\PY{l+s}{x2\PYZdq{}}\PY{p}{)}
\PY{n+nf}{print}\PY{p}{(}\PY{n}{best\PYZus{}sol}\PY{p}{)}
\PY{c+c1}{\PYZsh{} Check the value of objective function at optimal point}
\PY{n+nf}{print}\PY{p}{(}\PY{n+nf}{paste}\PY{p}{(}\PY{l+s}{\PYZdq{}}\PY{l+s}{Total cost: \PYZdq{}}\PY{p}{,}\PY{+w}{ }\PY{n}{optimum}\PY{o}{\PYZdl{}}\PY{n}{objval}\PY{p}{,}\PY{+w}{ }\PY{n}{sep}\PY{o}{=}\PY{l+s}{\PYZdq{}}\PY{l+s}{\PYZdq{}}\PY{p}{)}\PY{p}{)}
\end{Verbatim}
\end{tcolorbox}

    \begin{Verbatim}[commandchars=\\\{\}]
  x1   x2
1000  500
[1] "Total cost: 5500"
    \end{Verbatim}

    \begin{tcolorbox}[breakable, size=fbox, boxrule=1pt, pad at break*=1mm,colback=cellbackground, colframe=cellborder]
\prompt{In}{incolor}{ }{\boxspacing}
\begin{Verbatim}[commandchars=\\\{\}]

\end{Verbatim}
\end{tcolorbox}

    \section{Practical 02: \uppercase{Simplex Method with 2 variables using
Python}}\label{practical-02-simplex-method-with-2-variables-using-python}

\$ Max ~z=3x1+2x2\$

subject to

\(x1 + x2 <=4\)

\(x1 - x2 <=2\)

\(x1,x2>=0\)

    \begin{tcolorbox}[breakable, size=fbox, boxrule=1pt, pad at break*=1mm,colback=cellbackground, colframe=cellborder]
\prompt{In}{incolor}{ }{\boxspacing}
\begin{Verbatim}[commandchars=\\\{\}]
\PY{n}{from}\PY{+w}{ }\PY{n}{scipy.optimize}\PY{+w}{ }\PY{n}{import}\PY{+w}{ }\PY{n}{linprog}
\PY{c+c1}{\PYZsh{}Max z=3x1+2x2}
\PY{c+c1}{\PYZsh{}subject to}
\PY{c+c1}{\PYZsh{}x1 + x2 \PYZlt{}=4}
\PY{c+c1}{\PYZsh{}x1 \PYZhy{} x2 \PYZlt{}=2}
\PY{c+c1}{\PYZsh{}x1,x2\PYZgt{}=0}
\PY{n}{obj}\PY{+w}{ }\PY{o}{=}\PY{+w}{ }\PY{p}{[}\PY{l+m}{\PYZhy{}3}\PY{p}{,}\PY{+w}{ }\PY{l+m}{\PYZhy{}2}\PY{p}{]}
\PY{n}{lhs\PYZus{}ineq}\PY{+w}{ }\PY{o}{=}\PY{+w}{ }\PY{p}{[[}\PY{+w}{ }\PY{l+m}{1}\PY{p}{,}\PY{+w}{ }\PY{l+m}{1}\PY{p}{]}\PY{p}{,}\PY{+w}{ }\PY{c+c1}{\PYZsh{} Red constraint left side}
\PY{+w}{            }\PY{p}{[}\PY{l+m}{1}\PY{p}{,}\PY{+w}{ }\PY{l+m}{\PYZhy{}1}\PY{p}{]]}\PY{+w}{ }\PY{c+c1}{\PYZsh{} Blue constraint left side}
\PY{n}{rhs\PYZus{}ineq}\PY{+w}{ }\PY{o}{=}\PY{+w}{ }\PY{p}{[}\PY{l+m}{4}\PY{p}{,}\PY{+w}{ }\PY{l+m}{2}\PY{p}{]}\PY{+w}{ }\PY{c+c1}{\PYZsh{} Blue constraint right side}
\PY{n}{bnd}\PY{+w}{ }\PY{o}{=}\PY{+w}{ }\PY{p}{[}\PY{p}{(}\PY{l+m}{0}\PY{p}{,}\PY{+w}{ }\PY{n+nf}{float}\PY{p}{(}\PY{l+s}{\PYZdq{}}\PY{l+s}{inf\PYZdq{}}\PY{p}{)}\PY{p}{)}\PY{p}{,}\PY{+w}{ }\PY{c+c1}{\PYZsh{} Bounds of x}
\PY{+w}{       }\PY{p}{(}\PY{l+m}{0}\PY{p}{,}\PY{+w}{ }\PY{n+nf}{float}\PY{p}{(}\PY{l+s}{\PYZdq{}}\PY{l+s}{inf\PYZdq{}}\PY{p}{)}\PY{p}{)}\PY{p}{]}\PY{+w}{ }\PY{c+c1}{\PYZsh{} Bounds of y}
\PY{n}{opt}\PY{+w}{ }\PY{o}{=}\PY{+w}{ }\PY{n+nf}{linprog}\PY{p}{(}\PY{n}{c}\PY{o}{=}\PY{n}{obj}\PY{p}{,}\PY{+w}{ }\PY{n}{A\PYZus{}ub}\PY{o}{=}\PY{n}{lhs\PYZus{}ineq}\PY{p}{,}\PY{+w}{ }\PY{n}{b\PYZus{}ub}\PY{o}{=}\PY{n}{rhs\PYZus{}ineq}\PY{p}{,}
\PY{+w}{              }\PY{n}{bounds}\PY{o}{=}\PY{n}{bnd}\PY{p}{,}\PY{n}{method}\PY{o}{=}\PY{l+s}{\PYZdq{}}\PY{l+s}{revised simplex\PYZdq{}}\PY{p}{)}
\end{Verbatim}
\end{tcolorbox}

    \begin{Verbatim}[commandchars=\\\{\}]
<ipython-input-1-70a4ba51b253>:13: DeprecationWarning: `method='revised
simplex'` is deprecated and will be removed in SciPy 1.11.0. Please use one of
the HiGHS solvers (e.g. `method='highs'`) in new code.
  opt = linprog(c=obj, A\_ub=lhs\_ineq, b\_ub=rhs\_ineq,
    \end{Verbatim}

    \begin{tcolorbox}[breakable, size=fbox, boxrule=1pt, pad at break*=1mm,colback=cellbackground, colframe=cellborder]
\prompt{In}{incolor}{ }{\boxspacing}
\begin{Verbatim}[commandchars=\\\{\}]
\PY{n}{opt}
\end{Verbatim}
\end{tcolorbox}

            \begin{tcolorbox}[breakable, size=fbox, boxrule=.5pt, pad at break*=1mm, opacityfill=0]
\prompt{Out}{outcolor}{ }{\boxspacing}
\begin{Verbatim}[commandchars=\\\{\}]
 message: Optimization terminated successfully.
 success: True
  status: 0
     fun: -11.0
       x: [ 3.000e+00  1.000e+00]
     nit: 2
\end{Verbatim}
\end{tcolorbox}
        
    \begin{tcolorbox}[breakable, size=fbox, boxrule=1pt, pad at break*=1mm,colback=cellbackground, colframe=cellborder]
\prompt{In}{incolor}{ }{\boxspacing}
\begin{Verbatim}[commandchars=\\\{\}]
\PY{n}{opt.fun}
\end{Verbatim}
\end{tcolorbox}

            \begin{tcolorbox}[breakable, size=fbox, boxrule=.5pt, pad at break*=1mm, opacityfill=0]
\prompt{Out}{outcolor}{ }{\boxspacing}
\begin{Verbatim}[commandchars=\\\{\}]
-11.0
\end{Verbatim}
\end{tcolorbox}
        
    \begin{tcolorbox}[breakable, size=fbox, boxrule=1pt, pad at break*=1mm,colback=cellbackground, colframe=cellborder]
\prompt{In}{incolor}{ }{\boxspacing}
\begin{Verbatim}[commandchars=\\\{\}]
\PY{n}{opt.success}
\end{Verbatim}
\end{tcolorbox}

            \begin{tcolorbox}[breakable, size=fbox, boxrule=.5pt, pad at break*=1mm, opacityfill=0]
\prompt{Out}{outcolor}{ }{\boxspacing}
\begin{Verbatim}[commandchars=\\\{\}]
True
\end{Verbatim}
\end{tcolorbox}
        
    \begin{tcolorbox}[breakable, size=fbox, boxrule=1pt, pad at break*=1mm,colback=cellbackground, colframe=cellborder]
\prompt{In}{incolor}{ }{\boxspacing}
\begin{Verbatim}[commandchars=\\\{\}]
\PY{n}{opt.x}
\end{Verbatim}
\end{tcolorbox}

            \begin{tcolorbox}[breakable, size=fbox, boxrule=.5pt, pad at break*=1mm, opacityfill=0]
\prompt{Out}{outcolor}{ }{\boxspacing}
\begin{Verbatim}[commandchars=\\\{\}]
array([3., 1.])
\end{Verbatim}
\end{tcolorbox}
        
    \section{Practical 03: \uppercase {Simplex Method with 3 variables using
Python}} \label{practical-03-simplex-method-with-3-variables-using-python}

\(Min\ z= x1-3x2+2x3\)

subject to

\(3x1-x2+3x3<=7\)

\(-2x1+4x2<=12\)

\(-4x1+3x2+8x3<=10\)

\(x1,x2,x3>=0\)

    \begin{tcolorbox}[breakable, size=fbox, boxrule=1pt, pad at break*=1mm,colback=cellbackground, colframe=cellborder]
\prompt{In}{incolor}{ }{\boxspacing}
\begin{Verbatim}[commandchars=\\\{\}]
\PY{n}{from}\PY{+w}{ }\PY{n}{scipy.optimize}\PY{+w}{ }\PY{n}{import}\PY{+w}{ }\PY{n}{linprog}
\PY{c+c1}{\PYZsh{}Min z= x1\PYZhy{}3x2+2x3}
\PY{c+c1}{\PYZsh{}subject to}
\PY{c+c1}{\PYZsh{}3x1\PYZhy{}x2+3x3\PYZlt{}=7}
\PY{c+c1}{\PYZsh{}\PYZhy{}2x1+4x2\PYZlt{}=12}
\PY{c+c1}{\PYZsh{}\PYZhy{}4x1+3x2+8x3\PYZlt{}=10}
\PY{c+c1}{\PYZsh{}x1,x2,x3\PYZgt{}=0}
\PY{n}{obj}\PY{+w}{ }\PY{o}{=}\PY{+w}{ }\PY{p}{[}\PY{l+m}{1}\PY{p}{,}\PY{+w}{ }\PY{l+m}{\PYZhy{}3}\PY{p}{,}\PY{+w}{ }\PY{l+m}{2}\PY{p}{]}
\PY{n}{lhs\PYZus{}ineq}\PY{+w}{ }\PY{o}{=}\PY{+w}{ }\PY{p}{[[}\PY{+w}{ }\PY{l+m}{3}\PY{p}{,}\PY{+w}{ }\PY{l+m}{\PYZhy{}1}\PY{p}{,}\PY{+w}{ }\PY{l+m}{3}\PY{p}{]}\PY{p}{,}\PY{+w}{ }\PY{c+c1}{\PYZsh{} Red constraint left side}
\PY{+w}{            }\PY{p}{[}\PY{l+m}{\PYZhy{}2}\PY{p}{,}\PY{+w}{ }\PY{l+m}{4}\PY{p}{,}\PY{+w}{ }\PY{l+m}{0}\PY{p}{]}\PY{p}{,}\PY{+w}{ }\PY{c+c1}{\PYZsh{} Blue constraint left side}
\PY{+w}{            }\PY{p}{[}\PY{+w}{ }\PY{l+m}{\PYZhy{}4}\PY{p}{,}\PY{+w}{ }\PY{l+m}{3}\PY{p}{,}\PY{+w}{ }\PY{l+m}{8}\PY{p}{]]}\PY{+w}{ }\PY{c+c1}{\PYZsh{} Yellow constraint left side}
\end{Verbatim}
\end{tcolorbox}

    \begin{tcolorbox}[breakable, size=fbox, boxrule=1pt, pad at break*=1mm,colback=cellbackground, colframe=cellborder]
\prompt{In}{incolor}{ }{\boxspacing}
\begin{Verbatim}[commandchars=\\\{\}]
\PY{n}{rhs\PYZus{}ineq}\PY{+w}{ }\PY{o}{=}\PY{+w}{ }\PY{p}{[}\PY{l+m}{7}\PY{p}{,}\PY{+w}{ }\PY{c+c1}{\PYZsh{} Red constraint right side}
\PY{+w}{            }\PY{l+m}{12}\PY{p}{,}\PY{+w}{ }\PY{c+c1}{\PYZsh{} Blue constraint right side}
\PY{+w}{            }\PY{l+m}{10}\PY{p}{]}\PY{+w}{ }\PY{c+c1}{\PYZsh{} Yellow constraint right side}
\PY{n}{bnd}\PY{+w}{ }\PY{o}{=}\PY{+w}{ }\PY{p}{[}\PY{p}{(}\PY{l+m}{0}\PY{p}{,}\PY{+w}{ }\PY{n+nf}{float}\PY{p}{(}\PY{l+s}{\PYZdq{}}\PY{l+s}{inf\PYZdq{}}\PY{p}{)}\PY{p}{)}\PY{p}{,}\PY{+w}{ }\PY{c+c1}{\PYZsh{} Bounds of x}
\PY{+w}{       }\PY{p}{(}\PY{l+m}{0}\PY{p}{,}\PY{+w}{ }\PY{n+nf}{float}\PY{p}{(}\PY{l+s}{\PYZdq{}}\PY{l+s}{inf\PYZdq{}}\PY{p}{)}\PY{p}{)}\PY{p}{,}
\PY{+w}{       }\PY{p}{(}\PY{l+m}{0}\PY{p}{,}\PY{+w}{ }\PY{n+nf}{float}\PY{p}{(}\PY{l+s}{\PYZdq{}}\PY{l+s}{inf\PYZdq{}}\PY{p}{)}\PY{p}{)}\PY{p}{]}\PY{+w}{ }\PY{c+c1}{\PYZsh{} Bounds of y}
\end{Verbatim}
\end{tcolorbox}

    \begin{tcolorbox}[breakable, size=fbox, boxrule=1pt, pad at break*=1mm,colback=cellbackground, colframe=cellborder]
\prompt{In}{incolor}{ }{\boxspacing}
\begin{Verbatim}[commandchars=\\\{\}]
\PY{n}{opt}\PY{+w}{ }\PY{o}{=}\PY{+w}{ }\PY{n+nf}{linprog}\PY{p}{(}\PY{n}{c}\PY{o}{=}\PY{n}{obj}\PY{p}{,}\PY{+w}{ }\PY{n}{A\PYZus{}ub}\PY{o}{=}\PY{n}{lhs\PYZus{}ineq}\PY{p}{,}\PY{+w}{ }\PY{n}{b\PYZus{}ub}\PY{o}{=}\PY{n}{rhs\PYZus{}ineq}\PY{p}{,}
\PY{+w}{              }\PY{n}{bounds}\PY{o}{=}\PY{n}{bnd}\PY{p}{,}\PY{+w}{ }\PY{n}{method}\PY{o}{=}\PY{l+s}{\PYZdq{}}\PY{l+s}{revised simplex\PYZdq{}}\PY{p}{)}
\end{Verbatim}
\end{tcolorbox}

    \begin{Verbatim}[commandchars=\\\{\}]
<ipython-input-3-188681a0ccf8>:1: DeprecationWarning: `method='revised simplex'`
is deprecated and will be removed in SciPy 1.11.0. Please use one of the HiGHS
solvers (e.g. `method='highs'`) in new code.
  opt = linprog(c=obj, A\_ub=lhs\_ineq, b\_ub=rhs\_ineq,
    \end{Verbatim}

    \begin{tcolorbox}[breakable, size=fbox, boxrule=1pt, pad at break*=1mm,colback=cellbackground, colframe=cellborder]
\prompt{In}{incolor}{ }{\boxspacing}
\begin{Verbatim}[commandchars=\\\{\}]
\PY{n}{opt}
\end{Verbatim}
\end{tcolorbox}

            \begin{tcolorbox}[breakable, size=fbox, boxrule=.5pt, pad at break*=1mm, opacityfill=0]
\prompt{Out}{outcolor}{ }{\boxspacing}
\begin{Verbatim}[commandchars=\\\{\}]
 message: Optimization terminated successfully.
 success: True
  status: 0
     fun: -11.0
       x: [ 4.000e+00  5.000e+00  0.000e+00]
     nit: 2
\end{Verbatim}
\end{tcolorbox}
        
    \section{Practical 04: \uppercase{Simplex Method with Equality Constraints Using
Python}}\label{practical-04-simplex-method-with-equality-constraints-using-python}

\(Max \ z=x+2y\)

subject to

\(2x+y<=20\)

\$ -4x+5y\textless=10\$

\$ -x+2y\textgreater=-2\$

\$ -x+5y=15\$

\$ x,y\textgreater=0\$

    \begin{tcolorbox}[breakable, size=fbox, boxrule=1pt, pad at break*=1mm,colback=cellbackground, colframe=cellborder]
\prompt{In}{incolor}{ }{\boxspacing}
\begin{Verbatim}[commandchars=\\\{\}]
\PY{n}{from}\PY{+w}{ }\PY{n}{scipy.optimize}\PY{+w}{ }\PY{n}{import}\PY{+w}{ }\PY{n}{linprog}
\PY{c+c1}{\PYZsh{}Max z=x+2y}
\PY{c+c1}{\PYZsh{}subject to}
\PY{c+c1}{\PYZsh{}2x+y\PYZlt{}=20}
\PY{c+c1}{\PYZsh{}\PYZhy{}4x+5y\PYZlt{}=10}
\PY{c+c1}{\PYZsh{}\PYZhy{}x+2y\PYZgt{}=\PYZhy{}2}
\PY{c+c1}{\PYZsh{}\PYZhy{}x+5y=15}
\PY{c+c1}{\PYZsh{}x,y\PYZgt{}=0}

\PY{n}{obj}\PY{+w}{ }\PY{o}{=}\PY{+w}{ }\PY{p}{[}\PY{l+m}{\PYZhy{}1}\PY{p}{,}\PY{+w}{ }\PY{l+m}{\PYZhy{}2}\PY{p}{]}
\PY{n}{lhs\PYZus{}ineq}\PY{+w}{ }\PY{o}{=}\PY{+w}{ }\PY{p}{[[}\PY{+w}{ }\PY{l+m}{2}\PY{p}{,}\PY{+w}{ }\PY{l+m}{1}\PY{p}{]}\PY{p}{,}\PY{+w}{ }\PY{c+c1}{\PYZsh{} Red constraint left side}
\PY{+w}{            }\PY{p}{[}\PY{l+m}{\PYZhy{}4}\PY{p}{,}\PY{+w}{ }\PY{l+m}{5}\PY{p}{]}\PY{p}{,}\PY{+w}{ }\PY{c+c1}{\PYZsh{} Blue constraint left side}
\PY{+w}{            }\PY{p}{[}\PY{+w}{ }\PY{l+m}{1}\PY{p}{,}\PY{+w}{ }\PY{l+m}{\PYZhy{}2}\PY{p}{]]}\PY{+w}{ }\PY{c+c1}{\PYZsh{} Yellow constraint left side}
\PY{n}{rhs\PYZus{}ineq}\PY{+w}{ }\PY{o}{=}\PY{+w}{ }\PY{p}{[}\PY{l+m}{20}\PY{p}{,}\PY{+w}{ }\PY{c+c1}{\PYZsh{} Red constraint right side}
\PY{+w}{            }\PY{l+m}{10}\PY{p}{,}\PY{+w}{ }\PY{c+c1}{\PYZsh{} Blue constraint right side}
\PY{+w}{            }\PY{l+m}{2}\PY{p}{]}\PY{+w}{ }\PY{c+c1}{\PYZsh{} Yellow constraint right side}
\end{Verbatim}
\end{tcolorbox}

    \begin{tcolorbox}[breakable, size=fbox, boxrule=1pt, pad at break*=1mm,colback=cellbackground, colframe=cellborder]
\prompt{In}{incolor}{ }{\boxspacing}
\begin{Verbatim}[commandchars=\\\{\}]
\PY{n}{lhs\PYZus{}eq}\PY{+w}{ }\PY{o}{=}\PY{+w}{ }\PY{p}{[[}\PY{l+m}{\PYZhy{}1}\PY{p}{,}\PY{+w}{ }\PY{l+m}{5}\PY{p}{]]}\PY{+w}{ }\PY{c+c1}{\PYZsh{} Green constraint left side}
\PY{n}{rhs\PYZus{}eq}\PY{+w}{ }\PY{o}{=}\PY{+w}{ }\PY{p}{[}\PY{l+m}{15}\PY{p}{]}\PY{+w}{ }\PY{c+c1}{\PYZsh{} Green constraint right side}

\PY{n}{bnd}\PY{+w}{ }\PY{o}{=}\PY{+w}{ }\PY{p}{[}\PY{p}{(}\PY{l+m}{0}\PY{p}{,}\PY{+w}{ }\PY{n+nf}{float}\PY{p}{(}\PY{l+s}{\PYZdq{}}\PY{l+s}{inf\PYZdq{}}\PY{p}{)}\PY{p}{)}\PY{p}{,}\PY{+w}{ }\PY{c+c1}{\PYZsh{} Bounds of x}
\PY{+w}{       }\PY{p}{(}\PY{l+m}{0}\PY{p}{,}\PY{+w}{ }\PY{n+nf}{float}\PY{p}{(}\PY{l+s}{\PYZdq{}}\PY{l+s}{inf\PYZdq{}}\PY{p}{)}\PY{p}{)}\PY{p}{]}\PY{+w}{ }\PY{c+c1}{\PYZsh{} Bounds of y}
\end{Verbatim}
\end{tcolorbox}

    \begin{tcolorbox}[breakable, size=fbox, boxrule=1pt, pad at break*=1mm,colback=cellbackground, colframe=cellborder]
\prompt{In}{incolor}{ }{\boxspacing}
\begin{Verbatim}[commandchars=\\\{\}]
\PY{n}{opt}\PY{+w}{ }\PY{o}{=}\PY{+w}{ }\PY{n+nf}{linprog}\PY{p}{(}\PY{n}{c}\PY{o}{=}\PY{n}{obj}\PY{p}{,}\PY{+w}{ }\PY{n}{A\PYZus{}ub}\PY{o}{=}\PY{n}{lhs\PYZus{}ineq}\PY{p}{,}\PY{+w}{ }\PY{n}{b\PYZus{}ub}\PY{o}{=}\PY{n}{rhs\PYZus{}ineq}\PY{p}{,}
\PY{+w}{        }\PY{n}{A\PYZus{}eq}\PY{o}{=}\PY{n}{lhs\PYZus{}eq}\PY{p}{,}\PY{+w}{ }\PY{n}{b\PYZus{}eq}\PY{o}{=}\PY{n}{rhs\PYZus{}eq}\PY{p}{,}\PY{+w}{ }\PY{n}{bounds}\PY{o}{=}\PY{n}{bnd}\PY{p}{,}\PY{+w}{ }\PY{n}{method}\PY{o}{=}\PY{l+s}{\PYZdq{}}\PY{l+s}{revised simplex\PYZdq{}}\PY{p}{)}
\end{Verbatim}
\end{tcolorbox}

    \begin{Verbatim}[commandchars=\\\{\}]
<ipython-input-2-44a20d41c7cb>:1: DeprecationWarning: `method='revised simplex'`
is deprecated and will be removed in SciPy 1.11.0. Please use one of the HiGHS
solvers (e.g. `method='highs'`) in new code.
  opt = linprog(c=obj, A\_ub=lhs\_ineq, b\_ub=rhs\_ineq,
    \end{Verbatim}

    \begin{tcolorbox}[breakable, size=fbox, boxrule=1pt, pad at break*=1mm,colback=cellbackground, colframe=cellborder]
\prompt{In}{incolor}{ }{\boxspacing}
\begin{Verbatim}[commandchars=\\\{\}]
\PY{n}{opt}
\end{Verbatim}
\end{tcolorbox}

            \begin{tcolorbox}[breakable, size=fbox, boxrule=.5pt, pad at break*=1mm, opacityfill=0]
\prompt{Out}{outcolor}{ }{\boxspacing}
\begin{Verbatim}[commandchars=\\\{\}]
 message: Optimization terminated successfully.
 success: True
  status: 0
     fun: -16.818181818181817
       x: [ 7.727e+00  4.545e+00]
     nit: 3
\end{Verbatim}
\end{tcolorbox}
        
    \section{Practical 05: \uppercase {Solve Following linear programming problem using
Big M Simplex
method.}}\label{practical-05-solve-following-linear-programming-problem-using-big-m-simplex-method.}

Min \(z = 4x1 + x2\)

subjected to:

\(3x1 + 4x2 >= 20\)

\(x1 + 5x2 >= 15\)

\(x1, x2 >= 0\)

    \begin{tcolorbox}[breakable, size=fbox, boxrule=1pt, pad at break*=1mm,colback=cellbackground, colframe=cellborder]
\prompt{In}{incolor}{1}{\boxspacing}
\begin{Verbatim}[commandchars=\\\{\}]
\PY{n}{from}\PY{+w}{ }\PY{n}{scipy.optimize}\PY{+w}{ }\PY{n}{import}\PY{+w}{ }\PY{n}{linprog}

\PY{n}{obj}\PY{+w}{ }\PY{o}{=}\PY{+w}{ }\PY{p}{[}\PY{l+m}{4}\PY{p}{,}\PY{+w}{ }\PY{l+m}{1}\PY{p}{]}
\end{Verbatim}
\end{tcolorbox}

    \begin{tcolorbox}[breakable, size=fbox, boxrule=1pt, pad at break*=1mm,colback=cellbackground, colframe=cellborder]
\prompt{In}{incolor}{2}{\boxspacing}
\begin{Verbatim}[commandchars=\\\{\}]
\PY{n}{lhs\PYZus{}ineq}\PY{+w}{ }\PY{o}{=}\PY{+w}{ }\PY{p}{[[}\PY{+w}{ }\PY{l+m}{\PYZhy{}3}\PY{p}{,}\PY{+w}{ }\PY{l+m}{\PYZhy{}4}\PY{p}{]}\PY{p}{,}\PY{+w}{ }\PY{c+c1}{\PYZsh{} left side of first constraint}
\PY{+w}{            }\PY{p}{[}\PY{l+m}{\PYZhy{}1}\PY{p}{,}\PY{+w}{ }\PY{l+m}{\PYZhy{}5}\PY{p}{]]}\PY{+w}{ }\PY{c+c1}{\PYZsh{} right side of first constraint}
\PY{n}{rhs\PYZus{}ineq}\PY{+w}{ }\PY{o}{=}\PY{+w}{ }\PY{p}{[}\PY{l+m}{\PYZhy{}20}\PY{p}{,}\PY{+w}{ }\PY{c+c1}{\PYZsh{} right side of first constraint}
\PY{+w}{            }\PY{l+m}{\PYZhy{}15}\PY{p}{]}\PY{+w}{ }\PY{c+c1}{\PYZsh{} right side of Second constraint}
\end{Verbatim}
\end{tcolorbox}

    \begin{tcolorbox}[breakable, size=fbox, boxrule=1pt, pad at break*=1mm,colback=cellbackground, colframe=cellborder]
\prompt{In}{incolor}{3}{\boxspacing}
\begin{Verbatim}[commandchars=\\\{\}]
\PY{n}{bnd}\PY{+w}{ }\PY{o}{=}\PY{+w}{ }\PY{p}{[}\PY{p}{(}\PY{l+m}{0}\PY{p}{,}\PY{+w}{ }\PY{n+nf}{float}\PY{p}{(}\PY{l+s}{\PYZdq{}}\PY{l+s}{inf\PYZdq{}}\PY{p}{)}\PY{p}{)}\PY{p}{,}\PY{+w}{ }\PY{c+c1}{\PYZsh{} Bounds of x1}
\PY{+w}{       }\PY{p}{(}\PY{l+m}{0}\PY{p}{,}\PY{+w}{ }\PY{n+nf}{float}\PY{p}{(}\PY{l+s}{\PYZdq{}}\PY{l+s}{inf\PYZdq{}}\PY{p}{)}\PY{p}{)}\PY{p}{]}\PY{+w}{ }\PY{c+c1}{\PYZsh{} Bounds of x2}
\end{Verbatim}
\end{tcolorbox}

    \begin{tcolorbox}[breakable, size=fbox, boxrule=1pt, pad at break*=1mm,colback=cellbackground, colframe=cellborder]
\prompt{In}{incolor}{4}{\boxspacing}
\begin{Verbatim}[commandchars=\\\{\}]
\PY{n}{opt}\PY{+w}{ }\PY{o}{=}\PY{+w}{ }\PY{n+nf}{linprog}\PY{p}{(}\PY{n}{c}\PY{o}{=}\PY{n}{obj}\PY{p}{,}\PY{+w}{ }\PY{n}{A\PYZus{}ub}\PY{o}{=}\PY{n}{lhs\PYZus{}ineq}\PY{p}{,}\PY{+w}{ }\PY{n}{b\PYZus{}ub}\PY{o}{=}\PY{n}{rhs\PYZus{}ineq}\PY{p}{,}
\PY{+w}{              }\PY{n}{bounds}\PY{o}{=}\PY{n}{bnd}\PY{p}{,}\PY{n}{method}\PY{o}{=}\PY{l+s}{\PYZdq{}}\PY{l+s}{interior\PYZhy{}point\PYZdq{}}\PY{p}{)}
\end{Verbatim}
\end{tcolorbox}

    \begin{Verbatim}[commandchars=\\\{\}]
<ipython-input-4-42faacf79545>:1: DeprecationWarning: `method='interior-point'`
is deprecated and will be removed in SciPy 1.11.0. Please use one of the HiGHS
solvers (e.g. `method='highs'`) in new code.
  opt = linprog(c=obj, A\_ub=lhs\_ineq, b\_ub=rhs\_ineq,
    \end{Verbatim}

    \begin{tcolorbox}[breakable, size=fbox, boxrule=1pt, pad at break*=1mm,colback=cellbackground, colframe=cellborder]
\prompt{In}{incolor}{5}{\boxspacing}
\begin{Verbatim}[commandchars=\\\{\}]
\PY{n}{opt}
\end{Verbatim}
\end{tcolorbox}

            \begin{tcolorbox}[breakable, size=fbox, boxrule=.5pt, pad at break*=1mm, opacityfill=0]
\prompt{Out}{outcolor}{5}{\boxspacing}
\begin{Verbatim}[commandchars=\\\{\}]
 message: Optimization terminated successfully.
 success: True
  status: 0
     fun: 5.000000000236444
       x: [ 6.012e-11  5.000e+00]
     nit: 5
\end{Verbatim}
\end{tcolorbox}
        
    \section{Practical 06: RESOURCE ALLOCATION PROBLEM BY SIMPLEX
METHOD}\label{practical-06-resource-allocation-problem-by-simplex-method}

Use SciPy to solve the resource allocation problem stated as follows:

\(Max z= 20x1 + 12x2 +40x3 + 25x4\) \ldots\ldots\ldots\ldots.(profit)

subjected to:

\(x1 + x2 + x3 + x4 <= 50\) -------------(manpower)

\(3x1 + 2x2 + x3 <= 100\) -------------(material A)

\(x2 + 2x3 <= 90\) -------------(material B)

\(x1, x2, x3, x4 >= 0\)

    \begin{tcolorbox}[breakable, size=fbox, boxrule=1pt, pad at break*=1mm,colback=cellbackground, colframe=cellborder]
\prompt{In}{incolor}{1}{\boxspacing}
\begin{Verbatim}[commandchars=\\\{\}]
\PY{n}{from}\PY{+w}{ }\PY{n}{scipy.optimize}\PY{+w}{ }\PY{n}{import}\PY{+w}{ }\PY{n}{linprog}
\PY{n}{obj}\PY{+w}{ }\PY{o}{=}\PY{+w}{ }\PY{p}{[}\PY{l+m}{\PYZhy{}20}\PY{p}{,}\PY{+w}{ }\PY{l+m}{\PYZhy{}12}\PY{p}{,}\PY{+w}{ }\PY{l+m}{\PYZhy{}40}\PY{p}{,}\PY{+w}{ }\PY{l+m}{\PYZhy{}25}\PY{p}{]}\PY{+w}{ }\PY{c+c1}{\PYZsh{}profit objective function}
\end{Verbatim}
\end{tcolorbox}

    \begin{tcolorbox}[breakable, size=fbox, boxrule=1pt, pad at break*=1mm,colback=cellbackground, colframe=cellborder]
\prompt{In}{incolor}{2}{\boxspacing}
\begin{Verbatim}[commandchars=\\\{\}]
\PY{n}{lhs\PYZus{}ineq}\PY{+w}{ }\PY{o}{=}\PY{+w}{ }\PY{p}{[[}\PY{l+m}{1}\PY{p}{,}\PY{+w}{ }\PY{l+m}{1}\PY{p}{,}\PY{+w}{ }\PY{l+m}{1}\PY{p}{,}\PY{+w}{ }\PY{l+m}{1}\PY{p}{]}\PY{p}{,}\PY{+w}{ }\PY{c+c1}{\PYZsh{} Manpower}
\PY{+w}{            }\PY{p}{[}\PY{l+m}{3}\PY{p}{,}\PY{+w}{ }\PY{l+m}{2}\PY{p}{,}\PY{+w}{ }\PY{l+m}{1}\PY{p}{,}\PY{+w}{ }\PY{l+m}{0}\PY{p}{]}\PY{p}{,}\PY{+w}{ }\PY{c+c1}{\PYZsh{} Material A}
\PY{+w}{             }\PY{p}{[}\PY{l+m}{0}\PY{p}{,}\PY{+w}{ }\PY{l+m}{1}\PY{p}{,}\PY{+w}{ }\PY{l+m}{2}\PY{p}{,}\PY{+w}{ }\PY{l+m}{3}\PY{p}{]]}\PY{+w}{ }\PY{c+c1}{\PYZsh{} Material B}
\PY{n}{rhs\PYZus{}ineq}\PY{+w}{ }\PY{o}{=}\PY{+w}{ }\PY{p}{[}\PY{+w}{ }\PY{l+m}{50}\PY{p}{,}\PY{+w}{ }\PY{c+c1}{\PYZsh{} Manpower}
\PY{+w}{             }\PY{l+m}{100}\PY{p}{,}\PY{+w}{ }\PY{c+c1}{\PYZsh{} Material A}
\PY{+w}{             }\PY{l+m}{90}\PY{p}{]}\PY{+w}{ }\PY{c+c1}{\PYZsh{} Material B}
\end{Verbatim}
\end{tcolorbox}

    \begin{tcolorbox}[breakable, size=fbox, boxrule=1pt, pad at break*=1mm,colback=cellbackground, colframe=cellborder]
\prompt{In}{incolor}{3}{\boxspacing}
\begin{Verbatim}[commandchars=\\\{\}]
\PY{n}{opt}\PY{+w}{ }\PY{o}{=}\PY{+w}{ }\PY{n+nf}{linprog}\PY{p}{(}\PY{n}{c}\PY{o}{=}\PY{n}{obj}\PY{p}{,}\PY{+w}{ }\PY{n}{A\PYZus{}ub}\PY{o}{=}\PY{n}{lhs\PYZus{}ineq}\PY{p}{,}\PY{+w}{ }\PY{n}{b\PYZus{}ub}\PY{o}{=}\PY{n}{rhs\PYZus{}ineq}\PY{p}{,}
\PY{+w}{              }\PY{n}{method}\PY{o}{=}\PY{l+s}{\PYZdq{}}\PY{l+s}{revised simplex\PYZdq{}}\PY{p}{)}
\end{Verbatim}
\end{tcolorbox}

    \begin{Verbatim}[commandchars=\\\{\}]
<ipython-input-3-7085e87a9e94>:1: DeprecationWarning: `method='revised simplex'`
is deprecated and will be removed in SciPy 1.11.0. Please use one of the HiGHS
solvers (e.g. `method='highs'`) in new code.
  opt = linprog(c=obj, A\_ub=lhs\_ineq, b\_ub=rhs\_ineq,
    \end{Verbatim}

    \begin{tcolorbox}[breakable, size=fbox, boxrule=1pt, pad at break*=1mm,colback=cellbackground, colframe=cellborder]
\prompt{In}{incolor}{4}{\boxspacing}
\begin{Verbatim}[commandchars=\\\{\}]
\PY{n}{opt}
\end{Verbatim}
\end{tcolorbox}

            \begin{tcolorbox}[breakable, size=fbox, boxrule=.5pt, pad at break*=1mm, opacityfill=0]
\prompt{Out}{outcolor}{4}{\boxspacing}
\begin{Verbatim}[commandchars=\\\{\}]
 message: Optimization terminated successfully.
 success: True
  status: 0
     fun: -1900.0
       x: [ 5.000e+00  0.000e+00  4.500e+01  0.000e+00]
     nit: 2
\end{Verbatim}
\end{tcolorbox}
        
    \section{PRACTICAL 7: INFEASIBILITY IN SIMPLEX
METHOD}\label{practical-7-infeasibility-in-simplex-method}

\subsection{Solve following linear programming problem using simplex
method}\label{solve-following-linear-programming-problem-using-simplex-method}

\subsubsection{While solving linear programming problem using simplex
method, if one or more artificial variables remain in the basis at
positive level at the end of phase 1 computation , the problem has no
feasible solution( infeasible
solution).}\label{while-solving-linear-programming-problem-using-simplex-method-if-one-or-more-artificial-variables-remain-in-the-basis-at-positive-level-at-the-end-of-phase-1-computation-the-problem-has-no-feasible-solution-infeasible-solution.}

Example:

\$ Max ~z= 200x - 300y \$

subject to

\(2x+3y>=1200\)

\(x+y<=400\)

\(2x+3/2y>=900\)

\(x,y>=0\)

    \begin{tcolorbox}[breakable, size=fbox, boxrule=1pt, pad at break*=1mm,colback=cellbackground, colframe=cellborder]
\prompt{In}{incolor}{1}{\boxspacing}
\begin{Verbatim}[commandchars=\\\{\}]
\PY{n}{from}\PY{+w}{ }\PY{n}{scipy.optimize}\PY{+w}{ }\PY{n}{import}\PY{+w}{ }\PY{n}{linprog}
\PY{n}{obj}\PY{+w}{ }\PY{o}{=}\PY{+w}{ }\PY{p}{[}\PY{l+m}{\PYZhy{}200}\PY{p}{,}\PY{+w}{ }\PY{l+m}{300}\PY{p}{]}
\end{Verbatim}
\end{tcolorbox}

    \begin{tcolorbox}[breakable, size=fbox, boxrule=1pt, pad at break*=1mm,colback=cellbackground, colframe=cellborder]
\prompt{In}{incolor}{2}{\boxspacing}
\begin{Verbatim}[commandchars=\\\{\}]
\PY{n}{lhs\PYZus{}ineq}\PY{+w}{ }\PY{o}{=}\PY{+w}{ }\PY{p}{[[}\PY{+w}{ }\PY{l+m}{\PYZhy{}2}\PY{p}{,}\PY{+w}{ }\PY{l+m}{\PYZhy{}3}\PY{p}{]}\PY{p}{,}\PY{+w}{ }\PY{c+c1}{\PYZsh{} Red constraint left side}
\PY{+w}{            }\PY{p}{[}\PY{l+m}{1}\PY{p}{,}\PY{+w}{ }\PY{l+m}{1}\PY{p}{]}\PY{p}{,}\PY{+w}{ }\PY{c+c1}{\PYZsh{} Blue constraint left side}
\PY{+w}{            }\PY{p}{[}\PY{+w}{ }\PY{l+m}{\PYZhy{}2}\PY{p}{,}\PY{+w}{ }\PY{l+m}{\PYZhy{}1.5}\PY{p}{]]}\PY{+w}{ }\PY{c+c1}{\PYZsh{} Yellow constraint left side}
\PY{n}{rhs\PYZus{}ineq}\PY{+w}{ }\PY{o}{=}\PY{+w}{ }\PY{p}{[}\PY{l+m}{\PYZhy{}1200}\PY{p}{,}\PY{+w}{ }\PY{c+c1}{\PYZsh{} Red constraint right side}
\PY{+w}{              }\PY{l+m}{400}\PY{p}{,}\PY{+w}{ }\PY{c+c1}{\PYZsh{} Blue constraint right side}
\PY{+w}{            }\PY{l+m}{\PYZhy{}900}\PY{p}{]}\PY{+w}{ }\PY{c+c1}{\PYZsh{} Yellow constraint right side}
\PY{n}{bnd}\PY{+w}{ }\PY{o}{=}\PY{+w}{ }\PY{p}{[}\PY{p}{(}\PY{l+m}{0}\PY{p}{,}\PY{+w}{ }\PY{n+nf}{float}\PY{p}{(}\PY{l+s}{\PYZdq{}}\PY{l+s}{inf\PYZdq{}}\PY{p}{)}\PY{p}{)}\PY{p}{,}\PY{+w}{ }\PY{c+c1}{\PYZsh{} Bounds of x}
\PY{+w}{        }\PY{p}{(}\PY{l+m}{0}\PY{p}{,}\PY{+w}{ }\PY{n+nf}{float}\PY{p}{(}\PY{l+s}{\PYZdq{}}\PY{l+s}{inf\PYZdq{}}\PY{p}{)}\PY{p}{)}\PY{p}{]}\PY{+w}{ }\PY{c+c1}{\PYZsh{} Bounds of y}
\end{Verbatim}
\end{tcolorbox}

    \begin{tcolorbox}[breakable, size=fbox, boxrule=1pt, pad at break*=1mm,colback=cellbackground, colframe=cellborder]
\prompt{In}{incolor}{3}{\boxspacing}
\begin{Verbatim}[commandchars=\\\{\}]
\PY{n}{opt}\PY{+w}{ }\PY{o}{=}\PY{+w}{ }\PY{n+nf}{linprog}\PY{p}{(}\PY{n}{c}\PY{o}{=}\PY{n}{obj}\PY{p}{,}\PY{+w}{ }\PY{n}{A\PYZus{}ub}\PY{o}{=}\PY{n}{lhs\PYZus{}ineq}\PY{p}{,}\PY{+w}{ }\PY{n}{b\PYZus{}ub}\PY{o}{=}\PY{n}{rhs\PYZus{}ineq}\PY{p}{,}
\PY{+w}{              }\PY{n}{method}\PY{o}{=}\PY{l+s}{\PYZdq{}}\PY{l+s}{revised simplex\PYZdq{}}\PY{p}{)}
\end{Verbatim}
\end{tcolorbox}

    \begin{Verbatim}[commandchars=\\\{\}]
<ipython-input-3-7085e87a9e94>:1: DeprecationWarning: `method='revised simplex'`
is deprecated and will be removed in SciPy 1.11.0. Please use one of the HiGHS
solvers (e.g. `method='highs'`) in new code.
  opt = linprog(c=obj, A\_ub=lhs\_ineq, b\_ub=rhs\_ineq,
    \end{Verbatim}

    \begin{tcolorbox}[breakable, size=fbox, boxrule=1pt, pad at break*=1mm,colback=cellbackground, colframe=cellborder]
\prompt{In}{incolor}{4}{\boxspacing}
\begin{Verbatim}[commandchars=\\\{\}]
\PY{n}{opt}
\end{Verbatim}
\end{tcolorbox}

            \begin{tcolorbox}[breakable, size=fbox, boxrule=.5pt, pad at break*=1mm, opacityfill=0]
\prompt{Out}{outcolor}{4}{\boxspacing}
\begin{Verbatim}[commandchars=\\\{\}]
 message: The problem appears infeasible, as the phase one auxiliary problem
terminated successfully with a residual of 3.0e+02, greater than the tolerance
1e-12 required for the solution to be considered feasible. Consider increasing
the tolerance to be greater than 3.0e+02. If this tolerance is unnaceptably
large, the problem is likely infeasible.
 success: False
  status: 2
     fun: 120000.0
       x: [ 0.000e+00  4.000e+02]
     nit: 1
\end{Verbatim}
\end{tcolorbox}
        
    \section{PRACTICAL 8: DUAL SIMPLEX
METHOD}\label{practical-8-dual-simplex-method}

\subsection{Solve following linear programming problem using dual
simplex method using r
programming}\label{solve-following-linear-programming-problem-using-dual-simplex-method-using-r-programming}

\(Max \ z=40x1+50x2\)

subject to

\(2x1 + 3x2 <= 3\)

\(8x1 + 4x2 <= 5\)

\(x1, x2>=0\)

    \begin{tcolorbox}[breakable, size=fbox, boxrule=1pt, pad at break*=1mm,colback=cellbackground, colframe=cellborder]
\prompt{In}{incolor}{2}{\boxspacing}
\begin{Verbatim}[commandchars=\\\{\}]
\PY{n+nf}{install.packages}\PY{p}{(}\PY{l+s}{\PYZdq{}}\PY{l+s}{lpSolve\PYZdq{}}\PY{p}{)}
\end{Verbatim}
\end{tcolorbox}

    \begin{Verbatim}[commandchars=\\\{\}]
Installing package into ‘/usr/local/lib/R/site-library’
(as ‘lib’ is unspecified)

    \end{Verbatim}

    \begin{tcolorbox}[breakable, size=fbox, boxrule=1pt, pad at break*=1mm,colback=cellbackground, colframe=cellborder]
\prompt{In}{incolor}{3}{\boxspacing}
\begin{Verbatim}[commandchars=\\\{\}]
\PY{c+c1}{\PYZsh{} Import lpSolve package}
\PY{n+nf}{library}\PY{p}{(}\PY{n}{lpSolve}\PY{p}{)}
\PY{c+c1}{\PYZsh{} Set coefficients of the objective function}
\PY{n}{f.obj}\PY{+w}{ }\PY{o}{\PYZlt{}\PYZhy{}}\PY{+w}{ }\PY{n+nf}{c}\PY{p}{(}\PY{l+m}{40}\PY{p}{,}\PY{+w}{ }\PY{l+m}{50}\PY{p}{)}
\end{Verbatim}
\end{tcolorbox}

    \begin{tcolorbox}[breakable, size=fbox, boxrule=1pt, pad at break*=1mm,colback=cellbackground, colframe=cellborder]
\prompt{In}{incolor}{4}{\boxspacing}
\begin{Verbatim}[commandchars=\\\{\}]
\PY{c+c1}{\PYZsh{} Set matrix corresponding to coefficients of constraints by rows}
\PY{c+c1}{\PYZsh{} Do not consider the non\PYZhy{}negative constraint; it is automatically assumed}
\PY{n}{f.con}\PY{+w}{ }\PY{o}{\PYZlt{}\PYZhy{}}\PY{+w}{ }\PY{n+nf}{matrix}\PY{p}{(}\PY{n+nf}{c}\PY{p}{(}\PY{l+m}{2}\PY{p}{,}\PY{+w}{ }\PY{l+m}{3}\PY{p}{,}\PY{+w}{ }\PY{l+m}{8}\PY{p}{,}\PY{+w}{ }\PY{l+m}{4}\PY{p}{)}\PY{p}{,}\PY{+w}{ }\PY{n}{nrow}\PY{+w}{ }\PY{o}{=}\PY{+w}{ }\PY{l+m}{2}\PY{p}{,}\PY{+w}{ }\PY{n}{byrow}\PY{+w}{ }\PY{o}{=}\PY{+w}{ }\PY{k+kc}{TRUE}\PY{p}{)}
\end{Verbatim}
\end{tcolorbox}

    \begin{tcolorbox}[breakable, size=fbox, boxrule=1pt, pad at break*=1mm,colback=cellbackground, colframe=cellborder]
\prompt{In}{incolor}{5}{\boxspacing}
\begin{Verbatim}[commandchars=\\\{\}]
\PY{c+c1}{\PYZsh{} Set unequality signs}
\PY{n}{f.dir}\PY{+w}{ }\PY{o}{\PYZlt{}\PYZhy{}}\PY{+w}{ }\PY{n+nf}{c}\PY{p}{(}\PY{l+s}{\PYZdq{}}\PY{l+s}{\PYZlt{}=\PYZdq{}}\PY{p}{,}\PY{+w}{ }\PY{l+s}{\PYZdq{}}\PY{l+s}{\PYZlt{}=\PYZdq{}}\PY{p}{)}
\PY{c+c1}{\PYZsh{} Set right hand side coefficients}
\PY{n}{f.rhs}\PY{+w}{ }\PY{o}{\PYZlt{}\PYZhy{}}\PY{+w}{ }\PY{n+nf}{c}\PY{p}{(}\PY{l+m}{3}\PY{p}{,}\PY{+w}{ }\PY{l+m}{5}\PY{p}{)}
\end{Verbatim}
\end{tcolorbox}

    \begin{tcolorbox}[breakable, size=fbox, boxrule=1pt, pad at break*=1mm,colback=cellbackground, colframe=cellborder]
\prompt{In}{incolor}{7}{\boxspacing}
\begin{Verbatim}[commandchars=\\\{\}]
\PY{c+c1}{\PYZsh{} Final value (z)}
\PY{n+nf}{lp}\PY{p}{(}\PY{l+s}{\PYZdq{}}\PY{l+s}{max\PYZdq{}}\PY{p}{,}\PY{+w}{ }\PY{n}{f.obj}\PY{p}{,}\PY{+w}{ }\PY{n}{f.con}\PY{p}{,}\PY{+w}{ }\PY{n}{f.dir}\PY{p}{,}\PY{+w}{ }\PY{n}{f.rhs}\PY{p}{)}
\PY{c+c1}{\PYZsh{} Variables final values}
\PY{n+nf}{lp}\PY{p}{(}\PY{l+s}{\PYZdq{}}\PY{l+s}{max\PYZdq{}}\PY{p}{,}\PY{+w}{ }\PY{n}{f.obj}\PY{p}{,}\PY{+w}{ }\PY{n}{f.con}\PY{p}{,}\PY{+w}{ }\PY{n}{f.dir}\PY{p}{,}\PY{+w}{ }\PY{n}{f.rhs}\PY{p}{)}\PY{o}{\PYZdl{}}\PY{n}{solution}
\PY{c+c1}{\PYZsh{} Sensitivities}
\PY{n+nf}{lp}\PY{p}{(}\PY{l+s}{\PYZdq{}}\PY{l+s}{max\PYZdq{}}\PY{p}{,}\PY{+w}{ }\PY{n}{f.obj}\PY{p}{,}\PY{+w}{ }\PY{n}{f.con}\PY{p}{,}\PY{+w}{ }\PY{n}{f.dir}\PY{p}{,}\PY{+w}{ }\PY{n}{f.rhs}\PY{p}{,}\PY{+w}{ }\PY{n}{compute.sens}\PY{o}{=}\PY{k+kc}{TRUE}\PY{p}{)}\PY{o}{\PYZdl{}}\PY{n}{sens.coef.from}
\PY{n+nf}{lp}\PY{p}{(}\PY{l+s}{\PYZdq{}}\PY{l+s}{max\PYZdq{}}\PY{p}{,}\PY{+w}{ }\PY{n}{f.obj}\PY{p}{,}\PY{+w}{ }\PY{n}{f.con}\PY{p}{,}\PY{+w}{ }\PY{n}{f.dir}\PY{p}{,}\PY{+w}{ }\PY{n}{f.rhs}\PY{p}{,}\PY{+w}{ }\PY{n}{compute.sens}\PY{o}{=}\PY{k+kc}{TRUE}\PY{p}{)}\PY{o}{\PYZdl{}}\PY{n}{sens.coef.to}
\end{Verbatim}
\end{tcolorbox}

    
    \begin{Verbatim}[commandchars=\\\{\}]
Success: the objective function is 51.25 
    \end{Verbatim}

    
    \begin{enumerate*}
\item 0.1875
\item 0.875
\end{enumerate*}


    
    \begin{enumerate*}
\item 33.3333333333333
\item 20
\end{enumerate*}


    
    \begin{enumerate*}
\item 100
\item 60
\end{enumerate*}


    
    \begin{tcolorbox}[breakable, size=fbox, boxrule=1pt, pad at break*=1mm,colback=cellbackground, colframe=cellborder]
\prompt{In}{incolor}{8}{\boxspacing}
\begin{Verbatim}[commandchars=\\\{\}]
\PY{c+c1}{\PYZsh{} Dual Values (first dual of the constraints and then dual of the variables)}
\PY{c+c1}{\PYZsh{} Duals of the constraints and variables are mixed}
\PY{n+nf}{lp}\PY{p}{(}\PY{l+s}{\PYZdq{}}\PY{l+s}{max\PYZdq{}}\PY{p}{,}\PY{+w}{ }\PY{n}{f.obj}\PY{p}{,}\PY{+w}{ }\PY{n}{f.con}\PY{p}{,}\PY{+w}{ }\PY{n}{f.dir}\PY{p}{,}\PY{+w}{ }\PY{n}{f.rhs}\PY{p}{,}\PY{+w}{ }\PY{n}{compute.sens}\PY{o}{=}\PY{k+kc}{TRUE}\PY{p}{)}\PY{o}{\PYZdl{}}\PY{n}{duals}
\PY{c+c1}{\PYZsh{} Duals lower and upper limits}
\PY{n+nf}{lp}\PY{p}{(}\PY{l+s}{\PYZdq{}}\PY{l+s}{max\PYZdq{}}\PY{p}{,}\PY{+w}{ }\PY{n}{f.obj}\PY{p}{,}\PY{+w}{ }\PY{n}{f.con}\PY{p}{,}\PY{+w}{ }\PY{n}{f.dir}\PY{p}{,}\PY{+w}{ }\PY{n}{f.rhs}\PY{p}{,}\PY{+w}{ }\PY{n}{compute.sens}\PY{o}{=}\PY{k+kc}{TRUE}\PY{p}{)}\PY{o}{\PYZdl{}}\PY{n}{duals.from}
\PY{n+nf}{lp}\PY{p}{(}\PY{l+s}{\PYZdq{}}\PY{l+s}{max\PYZdq{}}\PY{p}{,}\PY{+w}{ }\PY{n}{f.obj}\PY{p}{,}\PY{+w}{ }\PY{n}{f.con}\PY{p}{,}\PY{+w}{ }\PY{n}{f.dir}\PY{p}{,}\PY{+w}{ }\PY{n}{f.rhs}\PY{p}{,}\PY{+w}{ }\PY{n}{compute.sens}\PY{o}{=}\PY{k+kc}{TRUE}\PY{p}{)}\PY{o}{\PYZdl{}}\PY{n}{duals.to}
\end{Verbatim}
\end{tcolorbox}

    \begin{enumerate*}
\item 15
\item 1.25
\item 0
\item 0
\end{enumerate*}


    
    \begin{enumerate*}
\item 1.25
\item 4
\item -1e+30
\item -1e+30
\end{enumerate*}


    
    \begin{enumerate*}
\item 3.75
\item 12
\item 1e+30
\item 1e+30
\end{enumerate*}


    
    \section{PRACTICAL 9: TRANSPORTATION
PROBLEM}\label{practical-9-transportation-problem}

\subsection{Solve following transportation problem in which cell entries
represent unit costs using R
programming.}\label{solve-following-transportation-problem-in-which-cell-entries-represent-unit-costs-using-r-programming.}

\begin{verbatim}
"Customer 1", "Customer 2", "Customer 3", "Customer 4" SUPPLY

sUPPLIER 1 10 2 20 11 15

sUPPLIER 1 12 7 9 20 25

sUPPLIER 1 4 14 16 18 10

DEMAND 5 15 15 15
\end{verbatim}

    \begin{tcolorbox}[breakable, size=fbox, boxrule=1pt, pad at break*=1mm,colback=cellbackground, colframe=cellborder]
\prompt{In}{incolor}{1}{\boxspacing}
\begin{Verbatim}[commandchars=\\\{\}]
\PY{n+nf}{install.packages}\PY{p}{(}\PY{l+s}{\PYZdq{}}\PY{l+s}{lpSolve\PYZdq{}}\PY{p}{)}
\end{Verbatim}
\end{tcolorbox}

    \begin{Verbatim}[commandchars=\\\{\}]
Installing package into ‘/usr/local/lib/R/site-library’
(as ‘lib’ is unspecified)

    \end{Verbatim}

    \begin{tcolorbox}[breakable, size=fbox, boxrule=1pt, pad at break*=1mm,colback=cellbackground, colframe=cellborder]
\prompt{In}{incolor}{2}{\boxspacing}
\begin{Verbatim}[commandchars=\\\{\}]
\PY{c+c1}{\PYZsh{} Import lpSolve package}
\PY{n+nf}{library}\PY{p}{(}\PY{n}{lpSolve}\PY{p}{)}
\PY{c+c1}{\PYZsh{} Set transportation costs matrix}
\PY{n}{costs}\PY{+w}{ }\PY{o}{\PYZlt{}\PYZhy{}}\PY{+w}{ }\PY{n+nf}{matrix}\PY{p}{(}\PY{n+nf}{c}\PY{p}{(}\PY{l+m}{10}\PY{p}{,}\PY{+w}{ }\PY{l+m}{2}\PY{p}{,}\PY{+w}{ }\PY{l+m}{20}\PY{p}{,}\PY{+w}{ }\PY{l+m}{11}\PY{p}{,}
\PY{+w}{                  }\PY{l+m}{12}\PY{p}{,}\PY{+w}{ }\PY{l+m}{7}\PY{p}{,}\PY{+w}{ }\PY{l+m}{9}\PY{p}{,}\PY{+w}{ }\PY{l+m}{20}\PY{p}{,}
\PY{+w}{                  }\PY{l+m}{4}\PY{p}{,}\PY{+w}{ }\PY{l+m}{14}\PY{+w}{ }\PY{p}{,}\PY{+w}{ }\PY{l+m}{16}\PY{p}{,}\PY{+w}{ }\PY{l+m}{18}\PY{p}{)}\PY{p}{,}\PY{+w}{ }\PY{n}{nrow}\PY{+w}{ }\PY{o}{=}\PY{+w}{ }\PY{l+m}{3}\PY{p}{,}\PY{+w}{ }\PY{n}{byrow}\PY{+w}{ }\PY{o}{=}\PY{+w}{ }\PY{k+kc}{TRUE}\PY{p}{)}

\PY{c+c1}{\PYZsh{} Set customers and suppliers\PYZsq{} names}
\PY{n+nf}{colnames}\PY{p}{(}\PY{n}{costs}\PY{p}{)}\PY{+w}{ }\PY{o}{\PYZlt{}\PYZhy{}}\PY{+w}{ }\PY{n+nf}{c}\PY{p}{(}\PY{l+s}{\PYZdq{}}\PY{l+s}{Customer 1\PYZdq{}}\PY{p}{,}\PY{+w}{ }\PY{l+s}{\PYZdq{}}\PY{l+s}{Customer 2\PYZdq{}}\PY{p}{,}\PY{+w}{ }\PY{l+s}{\PYZdq{}}\PY{l+s}{Customer 3\PYZdq{}}\PY{p}{,}\PY{+w}{ }\PY{l+s}{\PYZdq{}}\PY{l+s}{Customer 4\PYZdq{}}\PY{p}{)}
\PY{n+nf}{rownames}\PY{p}{(}\PY{n}{costs}\PY{p}{)}\PY{+w}{ }\PY{o}{\PYZlt{}\PYZhy{}}\PY{+w}{ }\PY{n+nf}{c}\PY{p}{(}\PY{l+s}{\PYZdq{}}\PY{l+s}{Supplier 1\PYZdq{}}\PY{p}{,}\PY{+w}{ }\PY{l+s}{\PYZdq{}}\PY{l+s}{Supplier 2\PYZdq{}}\PY{p}{,}\PY{+w}{ }\PY{l+s}{\PYZdq{}}\PY{l+s}{Supplier 3\PYZdq{}}\PY{p}{)}
\end{Verbatim}
\end{tcolorbox}

    \begin{tcolorbox}[breakable, size=fbox, boxrule=1pt, pad at break*=1mm,colback=cellbackground, colframe=cellborder]
\prompt{In}{incolor}{3}{\boxspacing}
\begin{Verbatim}[commandchars=\\\{\}]
\PY{c+c1}{\PYZsh{} Set unequality/equality signs for suppliers}
\PY{n}{row.signs}\PY{+w}{ }\PY{o}{\PYZlt{}\PYZhy{}}\PY{+w}{ }\PY{n+nf}{rep}\PY{p}{(}\PY{l+s}{\PYZdq{}}\PY{l+s}{\PYZlt{}=\PYZdq{}}\PY{p}{,}\PY{+w}{ }\PY{l+m}{3}\PY{p}{)}
\PY{c+c1}{\PYZsh{} Set right hand side coefficients for suppliers}
\PY{n}{row.rhs}\PY{+w}{ }\PY{o}{\PYZlt{}\PYZhy{}}\PY{+w}{ }\PY{n+nf}{c}\PY{p}{(}\PY{l+m}{15}\PY{p}{,}\PY{+w}{ }\PY{l+m}{25}\PY{p}{,}\PY{+w}{ }\PY{l+m}{10}\PY{p}{)}
\PY{c+c1}{\PYZsh{} Set unequality/equality signs for customers}
\PY{n}{col.signs}\PY{+w}{ }\PY{o}{\PYZlt{}\PYZhy{}}\PY{+w}{ }\PY{n+nf}{rep}\PY{p}{(}\PY{l+s}{\PYZdq{}}\PY{l+s}{\PYZgt{}=\PYZdq{}}\PY{p}{,}\PY{+w}{ }\PY{l+m}{4}\PY{p}{)}
\PY{c+c1}{\PYZsh{} Set right hand side coefficients for customers}
\PY{n}{col.rhs}\PY{+w}{ }\PY{o}{\PYZlt{}\PYZhy{}}\PY{+w}{ }\PY{n+nf}{c}\PY{p}{(}\PY{l+m}{5}\PY{p}{,}\PY{+w}{ }\PY{l+m}{15}\PY{p}{,}\PY{+w}{ }\PY{l+m}{15}\PY{p}{,}\PY{+w}{ }\PY{l+m}{15}\PY{p}{)}
\end{Verbatim}
\end{tcolorbox}

    \begin{tcolorbox}[breakable, size=fbox, boxrule=1pt, pad at break*=1mm,colback=cellbackground, colframe=cellborder]
\prompt{In}{incolor}{4}{\boxspacing}
\begin{Verbatim}[commandchars=\\\{\}]
\PY{c+c1}{\PYZsh{} Final value (z)}
\PY{n}{TotalCost}\PY{+w}{ }\PY{o}{\PYZlt{}\PYZhy{}}\PY{+w}{ }\PY{n+nf}{lp.transport}\PY{p}{(}\PY{n}{costs}\PY{p}{,}\PY{+w}{ }\PY{l+s}{\PYZdq{}}\PY{l+s}{min\PYZdq{}}\PY{p}{,}\PY{+w}{ }\PY{n}{row.signs}\PY{p}{,}\PY{+w}{ }\PY{n}{row.rhs}\PY{p}{,}\PY{+w}{ }\PY{n}{col.signs}\PY{p}{,}\PY{+w}{ }\PY{n}{col.rhs}\PY{p}{)}
\PY{c+c1}{\PYZsh{} Variables final values}
\PY{n+nf}{lp.transport}\PY{p}{(}\PY{n}{costs}\PY{p}{,}\PY{+w}{ }\PY{l+s}{\PYZdq{}}\PY{l+s}{min\PYZdq{}}\PY{p}{,}\PY{+w}{ }\PY{n}{row.signs}\PY{p}{,}\PY{+w}{ }\PY{n}{row.rhs}\PY{p}{,}\PY{+w}{ }\PY{n}{col.signs}\PY{p}{,}\PY{+w}{ }\PY{n}{col.rhs}\PY{p}{)}\PY{o}{\PYZdl{}}\PY{n}{solution}
\end{Verbatim}
\end{tcolorbox}

    A matrix: 3 × 4 of type dbl
\begin{tabular}{llll}
	 0 &  5 &  0 & 10\\
	 0 & 10 & 15 &  0\\
	 5 &  0 &  0 &  5\\
\end{tabular}


    
    \begin{Verbatim}[commandchars=\\\{\}]
Success: the objective function is 435
    \end{Verbatim}

    \begin{tcolorbox}[breakable, size=fbox, boxrule=1pt, pad at break*=1mm,colback=cellbackground, colframe=cellborder]
\prompt{In}{incolor}{5}{\boxspacing}
\begin{Verbatim}[commandchars=\\\{\}]
\PY{n+nf}{print}\PY{p}{(}\PY{n}{TotalCost}\PY{p}{)}
\end{Verbatim}
\end{tcolorbox}

    \begin{Verbatim}[commandchars=\\\{\}]
Success: the objective function is 435
    \end{Verbatim}

    \section{Practical 10: ASSIGNMENT
PROBLEM}\label{practical-10-assignment-problem}

\subsection{Solve following assignment problem represented in following
matrix using r
programming}\label{solve-following-assignment-problem-represented-in-following-matrix-using-r-programming}

\begin{verbatim}
   JOB1 JOB2 JOB3
W1  15   10    9
W2  9    15   10
W3  10   12    8
\end{verbatim}

    \begin{tcolorbox}[breakable, size=fbox, boxrule=1pt, pad at break*=1mm,colback=cellbackground, colframe=cellborder]
\prompt{In}{incolor}{1}{\boxspacing}
\begin{Verbatim}[commandchars=\\\{\}]
\PY{n+nf}{install.packages}\PY{p}{(}\PY{l+s}{\PYZdq{}}\PY{l+s}{lpSolve\PYZdq{}}\PY{p}{)}
\end{Verbatim}
\end{tcolorbox}

    \begin{Verbatim}[commandchars=\\\{\}]
Installing package into ‘/usr/local/lib/R/site-library’
(as ‘lib’ is unspecified)

    \end{Verbatim}

    \begin{tcolorbox}[breakable, size=fbox, boxrule=1pt, pad at break*=1mm,colback=cellbackground, colframe=cellborder]
\prompt{In}{incolor}{2}{\boxspacing}
\begin{Verbatim}[commandchars=\\\{\}]
\PY{c+c1}{\PYZsh{} Import lpSolve package}
\PY{n+nf}{library}\PY{p}{(}\PY{n}{lpSolve}\PY{p}{)}
\PY{c+c1}{\PYZsh{} Set assignment costs matrix}
\PY{n}{costs}\PY{+w}{ }\PY{o}{\PYZlt{}\PYZhy{}}\PY{+w}{ }\PY{n+nf}{matrix}\PY{p}{(}\PY{n+nf}{c}\PY{p}{(}\PY{l+m}{15}\PY{p}{,}\PY{+w}{ }\PY{l+m}{10}\PY{p}{,}\PY{+w}{ }\PY{l+m}{9}\PY{p}{,}
\PY{l+m}{9}\PY{p}{,}\PY{+w}{ }\PY{l+m}{15}\PY{p}{,}\PY{+w}{ }\PY{l+m}{10}\PY{p}{,}
\PY{l+m}{10}\PY{p}{,}\PY{+w}{ }\PY{l+m}{12}\PY{+w}{ }\PY{p}{,}\PY{l+m}{8}\PY{p}{)}\PY{p}{,}\PY{+w}{ }\PY{n}{nrow}\PY{+w}{ }\PY{o}{=}\PY{+w}{ }\PY{l+m}{3}\PY{p}{,}\PY{+w}{ }\PY{n}{byrow}\PY{+w}{ }\PY{o}{=}\PY{+w}{ }\PY{k+kc}{TRUE}\PY{p}{)}
\end{Verbatim}
\end{tcolorbox}

    \begin{tcolorbox}[breakable, size=fbox, boxrule=1pt, pad at break*=1mm,colback=cellbackground, colframe=cellborder]
\prompt{In}{incolor}{3}{\boxspacing}
\begin{Verbatim}[commandchars=\\\{\}]
\PY{c+c1}{\PYZsh{} Print assignment costs matrix}
\PY{n}{costs}
\end{Verbatim}
\end{tcolorbox}

    A matrix: 3 × 3 of type dbl
\begin{tabular}{lll}
	 15 & 10 &  9\\
	  9 & 15 & 10\\
	 10 & 12 &  8\\
\end{tabular}


    
    \begin{tcolorbox}[breakable, size=fbox, boxrule=1pt, pad at break*=1mm,colback=cellbackground, colframe=cellborder]
\prompt{In}{incolor}{4}{\boxspacing}
\begin{Verbatim}[commandchars=\\\{\}]
\PY{c+c1}{\PYZsh{} Final value (z)}
\PY{n+nf}{lp.assign}\PY{p}{(}\PY{n}{costs}\PY{p}{)}
\end{Verbatim}
\end{tcolorbox}

    
    \begin{Verbatim}[commandchars=\\\{\}]
Success: the objective function is 27 
    \end{Verbatim}

    
    \begin{tcolorbox}[breakable, size=fbox, boxrule=1pt, pad at break*=1mm,colback=cellbackground, colframe=cellborder]
\prompt{In}{incolor}{5}{\boxspacing}
\begin{Verbatim}[commandchars=\\\{\}]
\PY{c+c1}{\PYZsh{} Variables final values}
\PY{n+nf}{lp.assign}\PY{p}{(}\PY{n}{costs}\PY{p}{)}\PY{o}{\PYZdl{}}\PY{n}{solution}
\end{Verbatim}
\end{tcolorbox}

    A matrix: 3 × 3 of type dbl
\begin{tabular}{lll}
	 0 & 1 & 0\\
	 1 & 0 & 0\\
	 0 & 0 & 1\\
\end{tabular}


    

    % Add a bibliography block to the postdoc
    
    
    
\end{document}
